%
% $Id$
%

\chapter{Numerical solution of Markov models}  \label{SEC:Numerical}

\RELEASE{This chapter is still under revision.
More information can be obtained from the examples in
Chapter \ref{SEC:Examples} and from the various input files provided
with the {\smart} distribution.}


In this chapter, we consider systems whose underlying
stochastic process has a finite state space, and whose study can be carried
on through numerical methods.
Before describing the features implemented in {\smart} to solve such processes,
we discuss briefly a few key background notions in stochastic processes and
numerical linear algebra.

%        ===========================
\section{Discrete time Markov chains}
%        ===========================
A \index{discrete-time Markov chain (DTMC)}discrete-time Markov chain (DTMC)
is completely described by its transition
probability matrix $\P$ and its initial probability vector $\vp^{(0)}$,
where $\P[i,j]$ defines the probability
that a system in state $i$ will be in state $j$ at the next time step, and
$\vp^{(t)}[i]$ is the probability that the system is in state
$i$ at time $t \in \Naturals$.
Once $\P$ and $\vp^{(0)}$ have been specified explicitly or constructed from a
higher-level model, the following types of analysis are possible.
The interested reader is referred to \cite{Ross2003book} for more details.

%           =========================================
\subsection{Instantaneous transient analysis of DTMCs}\label{SEC:powerDTMC}
%           =========================================
The probability of being in state $j$ at time step $t \in \Naturals$
having started in state $i$ at time $0$ is the $(i,j)$-entry of
the $t\SUPTH$ power of the transition probability matrix, $\P^t[i,j]$.
However, powers of matrix $\P$ cannot be computed or stored in practice
due to the size of $\P$.
Thus, we make use of the knowledge of the initial probability vector
and compute instead the probability vector at time step $t$,
$\vp^{(t)} = \vp^{(0)} \cdot \P^t$, using the recursive relation
\[
  \vp^{(t+1)} = \vp^{(t)} \cdot \P.
\]

%           ======================================
\subsection{Cumulative transient analysis of DTMCs}
%           ======================================
If we are interested in the total time spent in each state during the first
$t \in \Naturals$ time steps, we compute
\[
   \vn^{(t)} = \sum_{y=0}^{t-1} \vp^{(y)}
\]
where $\vn^{(t)}[i]$ is the expected amount of time spent in state $i$
until time step $t$.

%           ==========================================
\subsection{Instantaneous stationary analysis of DTMCs}
%           ==========================================
Assuming that the DTMC is aperiodic\footnote{ The \emph{period} of
a state $i$ is the greatest common divisor of all time steps $t$
such that the probability of returning to state $i$ in $t$ time
steps, $\P^t[i,i]$, is nonzero. If a state has period greater than
one, it is said to be \index{periodic Markov chain}\emph{periodic},
otherwise it is \index{aperiodic Markov chain}\emph{aperiodic}.
The DTMC is aperiodic if all of its states are
aperiodic.}, {\smart} can compute the ``long-term'' probability of
being in each state. The method used in practice to compute the
\index{stationary probability distribution}stationary probability vector
$\vp = \lim_{t \to\infty} \vp^{(t)}$
depends on the properties of the DTMC.

\subsubsection*{Ergodic DTMCs}

\index{ergodic Markov chain}Ergodic DTMCs converge to a
\index{stationary probability distribution}stationary (invariant) probability distribution
independent of the initial state.  This is the case when a finite-state DTMC is aperiodic and
\index{irreducible Markov chain}irreducible\footnote{ A
Markov chain is irreducible if there is a path between every pair
of states.}.  Irreducibility on a finite state space
implies positive recurrence, which is necessary for a limiting distribution to exist.
Then, the stationary probability distribution (vector
$\vp$) uniquely satisfies $\vp = \vp \cdot \P$,
$\sum_{i}\vp[i]=1$.  The solution $\vp$ is strictly positive and
$\vp = \lim_{t \to\infty} \vp^{(t)}$.  In this case, {\smart} computes
$\vp$ by solving the homogeneous linear system
\[
  \vp \cdot (\P - \I) = \0 \ \text{ subject to }\ \sum_{i}\vp[i]=1.
\]

\subsubsection{\index{absorbing Markov chain}Absorbing DTMCs}

If the DTMC consists exclusively of transient and absorbing\footnote{
A state $i$ is \emph{transient} if there exists a state $j$ reachable from $i$
such that $i$ is not reachable from $j$.
A state $i$ is \emph{absorbing} if no other state is reachable from it,
that is, $\P[i,i] = 1$.
} states, we need to compute the
\index{stationary probability distribution}stationary probabilities only
for the absorbing states, since the transient states have zero probability.
However, these probabilities depend on the initial probability vector
$\vp^{(0)}$ in general.
Denoting the sets of absorbing and transient states by
$\aset$ and $\tset$, respectively, {\smart} computes $\vp$ as
\[
 \vp[\Set{A}] = \vp^{(0)}[\Set{A}] + \vn[\Set{T}] \cdot
 \P[\Set{T}, \Set{A}]
 ~~~~~~~~~~~~~~~~~~~~~~~~~~~~~~~~ \vp[\Set{T}] = \0
\]
where $\vn[\Set{T}]$ is described in \SectionRef{SEC:fundamental}.


\subsubsection{Arbitrary aperiodic DTMCs}

In the general case, we only require that all states of the DTMC are
\index{aperiodic Markov chain}aperiodic.
The DTMC can consist of transient states and classes of
recurrent (mutually reachable) states, and can be seen as an
\index{absorbing Markov chain}absorbing DTMC where
each absorbing state is replaced by an \index{ergodic Markov chain}ergodic DTMC.
The \index{stationary probability distribution}stationary probability vector $\vp$
is then computed by first computing the stationary probability vectors of each
\index{ergodic}ergodic class in isolation,
then computing the probability of being absorbed in each ergodic class, and
finally combining the stationary probability vectors of each ergodic class
using the absorption probabilities as weights (of course, the probability
of transient states is zero in this case as well).

\subsubsection{Periodic DTMCs}

For \index{periodic Markov chain}periodic DTMCs with period $\delta > 1$, $\P^t[i, j]$ does not
have a limit as $t\to\infty$ except for $\P^t[i, j]\to 0$ when all
state are null.  However, the limits of $\P^{t\delta+u}[i, j]$
exist as $t\to\infty$ but depend on the initial state $i$.
Although there exists a unique solution satisfying $$\vp = \vp
\cdot \P,\quad \sum_{i}\vp[i]=\delta,$$ {\smart} does not check
explicitly that the DTMC is periodic, much less compute the
period $\delta$ needed to find the solution. Nevertheless,
numerical methods might still give the correct answer, but in a
time-averaged sense: the computed vector $\vp$ satisfies $\vp =
\lim_{t \to\infty} \vn^{(t)}/t$.

If the (iterative) solution methods are unable to compute a
solution, it is simply because convergence was not obtained within
the maximum number of iterations allowed, and a warning is issued
in this case.  In other words, periodicity does not lead to
incorrect answers in {\smart}.


%           =======================================
\subsection{Cumulative stationary analysis of DTMCs}\label{SEC:fundamental}
%           =======================================
We can also consider the (expected) total time spent in each state
by computing the vector $\vn = \lim_{t \to\infty} \vn^{(t)}$. Of
course, $\vn[i] = \infty$ for any state $i$ with a positive
\index{stationary probability distribution}stationary probability,
thus the entries in $\vn$ are meaningful
only for the transient states. In this case, $\vn[i]$ is the
expected time spent in transient state $i$ before the DTMC reaches
an absorbing state, or a class of \index{ergodic}ergodic states. The portion of
$\vn$ corresponding to the transient states is the solution of the
non-homogeneous linear system
\[
  \vn[\Set{T}] \cdot (\P[\Set{T}, \Set{T}] - \I)
  = -\vp^{(0)}[\Set{T}]
\]
where $\P[\Set{T}, \Set{T}]$ is the submatrix of $\P$
corresponding to transitions between transient states.

%        =============================
\section{Continuous time Markov chains}
%        =============================
A \index{continuous-time Markov chain (CTMC)}continuous-time Markov chain (CTMC)
is completely described by its transition
rate matrix $\R$ and its initial probability vector
$\gvect{\pi}^{(0)}$, where $\R[i,j]$ defines the rate of going to state
$j$ given that the chain is in state $i$, and
$\gvect{\pi}^{(t)}[i]$ is the probability that the system is
in state $i$ at time $t \geq 0$.
A transition from a state to itself is meaningless in a CTMC,
thus, the diagonal elements of $\R$ are zero.
The infinitesimal generator matrix, $\Q$, is identical to the matrix $\R$
except for the diagonal elements, which are set so that each row of
$\Q$ sums to zero: $\Q[i,i] = - \sum_{j \neq i} \R[i,j]$.
Once $\Q$ (or $\R$) and $\gvect{\pi}^{(0)}$ have been specified explicitly
or constructed from a higher-level model, the following types of analysis
are possible.
The interested reader is referred to \cite{1993IMA-SRNs,Stewart1994book}
for more details.

%           =========================================
\subsection{Instantaneous transient analysis of CTMCs}\label{SEC:inst-trans-anal-CTMC}
%           =========================================
To compute $\gvect{\pi}(t)$, the probability vector at time $t$, we use a
technique called \index{uniformization (randomization)}\emph{uniformization}
(or \emph{randomization})
\cite{Grassmann1991}, which constructs the ``uniformized DTMC'' defined by
\[
   \P = \frac{\Q}{q} + \I, \quad \vp^{(0)} = \vpi(0)
\]
where $q$ satisfies $q \ge \max_i\{-\Q[i,i]\}$.  Observe that with this
choice for $q$, the CTMC will be sampled at a rate sufficient to observe the
fastest CTMC state transitions.  This means that the DTMC is constructed such that
states that are slower to transition to another state will have self loops.
Consequently, the DTMC is guaranteed to be aperiodic.
The probability vector $\gvect{\pi}(t)$ can then be written as
\[
  \gvect{\pi}(t) = \sum_{n=0}^{\infty} \frac{e^{-qt}(qt)^n}{n!}
     \cdot \vp^{(n)}
\]
where ${e^{-qt}(qt)^n}/{n!}$ is the Poisson probability of
observing $n$ state transitions of the CTMC ($n$ jumps of the DTMC) within time $t$,
and $\vp^{(n)}$ is the probability distribution of state occupancy after $n$ jumps.
Given $\P$, the vector $\vp^{(n)}$ is computed using the power method as discussed
in \SectionRef{SEC:powerDTMC}. {\smart} computes the Poisson probabilities using the
numerically stable Fox-Glynn algorithm \cite{Fox1988} and computes the summation
only up to a large-enough value $N$.
Since the absolute error of each element of $\gvect{\pi}(t)$
is bounded by the complement of the elements of the Poisson pmf summed up to $N$:
\[
  \overline{p}(N) = 1 - \sum_{n=0}^{N} \frac{e^{-qt}(qt)^n}{n!}.
\]
{\smart} determines the value of $N$ \emph{a priori} such that $N$ is the
smallest integer where $\overline{p}(N) \le \epsilon$
for a desired absolute precision $\epsilon$, which is given by the
\index{NumericalPrecision option}\Code{NumericalPrecision} option.

%           ======================================
\subsection{Cumulative transient analysis of CTMCs}\label{SEC:cumul-trans-anal-CTMC}
%           ======================================
{\smart} uses \index{uniformization (randomization)}uniformization also to compute $\vsigma(t) = \int_0^t
\vpi(u) du$, the cumulative sojourn time vector up to time $t$, or in other words,
the total amount of time spent in each state during the interval $(0,t)$.
Using integration by parts, $\gvect{\sigma}(t)$ can be reformulated as
$$
  \vsigma(t)
  = \int_{0}^{t}
    \sum_{n=0}^{\infty} \frac{e^{-q u} (q u)^n}{n!} \vp^{[n]} d u
  = \sum_{n=0}^{\infty} \vect{p}^{[n]}
    \int_{0}^{t} \frac{e^{-q u} (q u)^n}{n!} d u
  = \frac{1}{q} \sum_{n=0}^{\infty} \vp^{(n)} \sum_{k=n+1}^{\infty}
  \frac{e^{-q t} (q t)^k}{k!} =
  \frac{1}{q} \sum_{n=0}^{\infty} \vp^{(n)} \overline{p}(n).
$$
The summation is computed only up to a large-enough value $N'$ that satisfies
\[
  N' = \min_{k}\left\{t - \frac{1}{q} \sum_{n=0}^{k} \overline{p}(n) \right\} \leq \epsilon
\]
where, again, $\epsilon$ is given by the
\index{NumericalPrecision option}\Code{NumericalPrecision} option.
Because the complexity lies with the computation of $\vp^{(n)}$,
{\smart} automatically computes $\vsigma(t)$ at the same time $\gvect{\pi}(t)$
is computed.  Consequently, the actual value, $N$ or $N'$, that truncates
the summations is $\max\{N, N'\}$.

%           ======================
\subsection{Steady-state detection}
%           ======================
When the boolean option \index{SSdetect option}\Code{SSdetect} = \Code{true},
which is the default, {\smart} checks for steady-state conditions when performing
the above transient analysis just in case the CTMC has a limiting distribution,
$\gvect{\pi}=\lim_{u\to\infty} \gvect{\pi}(u)$, which may be dependent on the initial state.
If the solution time $t$ is large enough then $\vp^{(n)}$ may converge to a limiting distribution
as $n$ becomes large and before reaching the truncation point given by $\max\{N, N'\}$.
{\smart} assumes steady-state conditions when the relative error between successive iterates,
$\vp^{(n)}$ and $\vp^{(n+1)}$, is less than or equal to \Code{NumericalPrecision} times a
scaling factor given by \Code{SSsens} $\in(0, 1)$.
The scaling factor introduces a margin of error between the
absolute relative ``distance'' $\|\vp^{(n)}-\vp^{(n+1)}\|$ and
the continuous-time solution $\gvect{\pi}(t)$ given when $\vp^{(n)}$ is subordinated
to the Poisson process; that is, we want the error in the latter to dominate.
The scaling factor \index{SSsens option}\Code{SSsens}
is a user-defined option with a default value of $0.25$.

%           ==========================================
\subsection{Instantaneous stationary analysis of CTMCs}
%           ==========================================
Analogous to the discrete case, we can consider the ``long-term'' probability
of being in each state of a CTMC by computing the
\index{stationary probability distribution}stationary probability vector
\[
\gvect{\pi} = \lim_{t \rightarrow \infty} \gvect{\pi}(t).
\]
Like DTMCs, analysis of a general CTMC requires the analysis of many
\index{ergodic Markov chain}ergodic CTMCs and an \index{absorbing Markov chain}absorbing CTMC.

\subsubsection{Ergodic CTMCs}

Like ergodic DTMCs, \index{ergodic Markov chain}ergodic CTMCs also converge to a
\index{stationary probability distribution}stationary probability distribution
independent of the initial state.
In this case, a finite-state CTMC need only be
\index{irreducible Markov chain}irreducible since periodicity is not an issue
in continuous time.  Again, irreducibility on a finite state space
implies positive recurrence, which is necessary for a limiting distribution to exist.
%An ergodic CTMC has the property that for all pairs of states $i$ and $j$ in
%the CTMC, $j$ is reachable from $i$.
Then, the stationary probability vector, $\gvect{\pi}$, uniquely satisfies
\[
  \gvect{\pi} \cdot \Q = \0 \ \text{ subject to }\ \sum_{i} \gvect{\pi}[i] = 1.
\]


\MSG{New from here.}
Currently, {\smart} supports several different numerical algorithm to calculate the steady-state solutions
for ergodic CTMCs, which is set by option \emph{SolutionType} and \emph{Solver}.

\emph{SolutionType} has following options:
\begin{itemize}
\item NUMERICAL
\item APPROXIMATE\_EVMDD
\item EXACT\_EVMDD
\item SIMULATION
\end{itemize}

\emph{Solver} has following options:
\begin{itemize}
\item POWER
\item JACOBI
\item GAUSS\_SEIDEL
\end{itemize}

For example, {\smart} will utilize symbolic engine and Jacobi iteration by the following options:

\begin{verbatim}
# StateStorage MDD_SATURATION_LEVEL
# SolutionType EXACT_EVMDD
# Solver JACOBI
\end{verbatim}


\subsubsection{\index{absorbing Markov chain}Absorbing CTMCs}

If the CTMC consists only of transient and absorbing states,
we can compute the \index{stationary probability distribution}stationary probabilities
for the absorbing states by \emph{embedding} the CTMC, obtaining the DTMC
\[
  \P[i,j] = \left\{ \begin{array}{ll} %}
             \frac{\Q[i,j]}{-\Q[i,i]}
             &
             \mbox{if $i\neq j$}
             \\
             0
             &
             \mbox{otherwise}
                  \end{array} \right.
\]
with $\vp^{(0)} = \gvect{\pi}(0)$.
We then compute $\gvect{\pi} = \vp$ in a way analogous to the
discrete case.

%           =======================================
\subsection{Cumulative stationary analysis of CTMCs}
%           =======================================
Analogous the the DTMC case, we can compute the total expected time spent in
each state of a CTMC as the limit
\[
\gvect{\sigma} = \lim_{t \rightarrow \infty} \vsigma(t).
\]
For transient states, the entries in $\gvect{\sigma}$ are finite,
and represent the expected time spent in each transient state before
the CTMC reaches an absorbing state (or a class of \index{ergodic}ergodic states).
The portion of $\gvect{\sigma}$ corresponding to the transient states is the
solution to
\[
  \gvect{\sigma}[\Set{T}] \cdot \Q[\Set{T}, \Set{T}]
  = -\gvect{\pi}(0)[\Set{T}]
\]
where $\Q[\Set{T}, \Set{T}]$ is the submatrix of $\Q$
corresponding to transitions from transient states to transient states.

%        ===========================
\section{Semi-regenerative processes}\label{SEC:semi-regen-processes}
%        ===========================
While more complex than simple Markov chains,
\index{semi-regenerative process}semi-regenerative processes may still be analyzed using
the traditional Markov chain methods discussed previously, but only under certain conditions.
Essentially, given a discrete-state, discrete-event model where the events have \emph{both}
discrete and continuous \index{phase-type distribution}phase-type distributed delays
(\index{ph int}\Code{ph\! int} and \index{ph real}\Code{ph\! real}, respectively),
the underlying process will be semi-regenerative
if the \Code{ph\! int} events are always \emph{synchronized} when active.
By \index{synchronization property}synchronized,
we mean that the phase changes of all active \Code{ph\! int} events occur at precisely the
same (discrete) time when active.
Then, the regeneration times can be defined by \Code{ph\! int} events
(simple phase changes or other state changes triggered by the event).
The only other active events that can trigger state changes between successive regeneration times
are of \Code{ph\! real} type.  Consequently, the subordinate processes that evolve between
regeneration times are simple CTMCs, and, as is always the case for semi-regenerative processes,
the embedded \index{Markov renewal process}Markov renewal process gives rise to an embedded DTMC.
Semi-regenerative processes encountered in this way can be decomposed into
DTMC and CTMC subproblems; however, because the DTMC and CTMC ``submodels'' interact
with each other, they must be solved through a concerted effort.
Note that the underlying process may still be semi-regenerative without the synchronization
assumption, but without a decomposition into Markov chain subproblems, numerical solution
may not be practical.

At present, {\smart} is capable of numerically computing the stationary probability distribution
of a semi-regenerative process with the synchronization assumption when
the embedded DTMC is \index{ergodic}ergodic with a stationary solution.
Synchronization among \Code{ph\! int} events is automatically checked while generating the
underlying state space.  A warning is issued to the user if the requisite synchronization
assumption is not present, which precludes a numerical solution.
In what follows, we will only present the key formulas for computing stationary solutions
and refer the interested reader to \cite{Jones-NASA-00}
for a discourse on this subject and a detailed presentation of the solution algorithms.

Let $\P$ be the transition matrix of the embedded DTMC with state space $\mathcal{E}$
that evolves at each regeneration time.
The matrix $\P$ is computed one row at a time by exploring from each known state $i\in\mathcal{E}$
and constructing the subordinate CTMC that evolves until the next regeneration time is reached
with new embedded states $j\in\mathcal{E}$ such that $\P[i, j]>0$.
Actually, only values $\P[i, j] > \Code{Pruning}$ are stored where the
user option \index{Pruning option}\Code{Pruning} is
considered ``small enough'' with a default value of $10^{-20}$.
Let $\Q_i$ be the infinitesimal generator matrix of the subordinate CTMC on the state space
$\mathcal{S}_i$ originating from embedded state $i$.  Take the basic step of all \Code{ph\! int}
to be unity.  Then, when both \Code{ph\! int} and \Code{ph\! real} events are active,
the entries of matrix $\P$ can be computed from
\[
  \P[i, j] = \sum_{k\in\mathcal{S}_i} e^{\Q_i}[i, k] \matr{\Delta}[k, j]
\]
where the matrix $e^{\Q_i}$ is the (matrix exponential) solution of the CTMC at time one,
which is actually computed using the \index{uniformization (randomization)}uniformization
procedure discussed in Sections \ref{SEC:inst-trans-anal-CTMC} and \ref{SEC:cumul-trans-anal-CTMC}.
The matrix $\matr{\Delta}[k, j]$ gives the probability of jumping from state $k$ to state $j$
at the regeneration time and is computed automatically from the model specification.
When only \Code{ph\! int} or only \Code{ph\! real} events are active,
a regeneration occurs after each state transition due to the strong Markov property, and we
simply replace $e^{\Q_i}[i, k]$ in the equation above with the indicator function $1\{i=k\}$.
Conditioned on each $i\in\mathcal{E}$, let the matrix $\gvect{\sigma}$ be the expected sojourn time
in each state $k\in\mathcal{S}_i$ until the next regeneration time, which is computed from
\[
  \gvect{\sigma}[i, k] =
  \begin{cases}
    \int_{0}^{1} e^{\Q_i u}[i, k]\,du
                               & \text{if both \Code{ph\! int} and \Code{ph\! real} are active}, \\
    1\{i=k\} (-Q_i[i, i])^{-1} & \text{if only \Code{ph\! real} events are active}, \\
    1\{i=k\}                   & \text{if only \Code{ph\! int} events are active},  \\
    0                          & \text{otherwise}.
  \end{cases}
\]
The computation in the first case can be performed at the same time $e^{\Q_i}$ is computed.
After constructing the matrix $\P$,
the stationary solution, $\vp$, of the embedded DTMC is computed from $\vp \cdot (\P - \I) = \0$.
The whole state space for the semi-regenerative process is defined by
$\mathcal{S}\equiv\bigcup_{i\in\mathcal{E}} \mathcal{S}_i$.
Finally, the stationary solution, $\gvect{\pi}$, of the semi-regenerative process is computed by
considering the average time spent in any state between regeneration times
(conditioned on each embedded state),
\[
  \gvect{\gamma}[k] = \sum_{i\in\mathcal{E}} \vp[i]\cdot\gvect{\sigma}[i, k],
\]
and then normalizing,
\[
  \gvect{\pi}[k] = \frac{\gvect{\gamma}[k]}{\sum_{j} \gvect{\gamma}[j]},
\]
to obtain the proportion of time spent in any state between regeneration times
for each state $k\in\mathcal{S}$.

\PRIVATE{
{\smart} actually avoids having to compute and store the entire matrix
$\gvect{\sigma}\in\Reals^{|\mathcal{E}|\times|\mathcal{S}|}$ by being ``measure driven.''
That is, $\gvect{\sigma}[i, k]$ for each state $i\in\mathcal{E}$ in turn is used to
``distill'' the user-defined reward functions $f : \mathcal{S}\to\Reals$
(used to compute measures) into a new reward functions $\tilde{f} : \mathcal{E}\to\Reals$,
\[
  \tilde{f}(i) = \sum_{k\in\mathcal{S}_i} \gvect{\sigma}[i, k]\cdot f(k),
\]
defined only on the embedded state space $\mathcal{E}$, not the whole state space $\mathcal{S}$.
This way, $\gvect{\sigma}[i, \cdot]$ can be discarded once it is used to compute $\tilde{f}(i)$.
}

We have just described the \index{embedding strategies}``embedding strategy'' used when the
option \index{EmbedWithElim option}\Code{EmbedWithElim} = 0 is selected, which is the default.
Alternative embedding strategies, which attempt to reduce the size of the state space $\mathcal{E}$,
are also available that exploit the fact that oftentimes more than one embedded Markov process is
present \cite{2001PNPM-PDPN}.
Essentially, embedded states can be eliminated by being selective in our choice of
which embedded Markov process to observe.  At present, there are four possible
embedding strategies to choose from:
\begin{table}[H]
\centering
\begin{tabular}{|c|p{4.25in}|}
\hline
  \Code{EmbedWithElim} & \textbf{Description} \\
  \hline
  \hline
  0 & most basic, the one described above, 0 embedded state are eliminated \\ \hline
  1 & eliminate embedded states that arise when only
      \Code{ph\! int} or only \Code{ph\! real} events are active \\ \hline
  2 & eliminate embedded states that only record phase changes
      when both \Code{ph\! int} and \Code{ph\! real} events are active \\ \hline
  3 & 1+2=3, use both 1 and 2 together \\
  \hline
\end{tabular}
\end{table}

Another user option called
\index{Epsilon option}\Code{Epsilon} comes into play when using embedding strategies 2 or 3.
Embedding strategy 2 (and hence 3) tries to find \emph{all possible} regeneration times that:
\begin{itemize}
  \item comprise a single step in $\P$,
  \item are (mostly) due to synchronized events with \Code{ph\! int} delays, and
  \item it is certain that the event will trigger a state change where more than just the
        phase component changes, such as a change in the marking in the case of \Code{spn} models.
\end{itemize}
Because {\smart} will try to consider the entire (pmf) support of active \Code{ph\! int} events
in search of regeneration times with all of the above properties, a truncation point must be
introduced in case the support is infinite in length, such as the infinite support of the
geometric distribution.
The user option \index{Epsilon option}\Code{Epsilon} $\in$ $(0, 1)$ provides a truncation
point by stopping the search when the smallest pmf tail of any active \Code{ph\! int} event
becomes less than or equal to \Code{Epsilon}. \Code{Epsilon} has a default value of $10^{-10}$.

Finally, the following print options are provided that
take boolean values $\{\Code{true}, \Code{false}\}$,
all \Code{false} by default:
\index{ShowEMCdata, ShowEMCprob, ShowEMCsize, ShowSpecial options}
\begin{table}[H]
\centering
\begin{tabular}{|c|p{4.25in}|}
\hline
  \textbf{Option} & \textbf{Prints the following if \Code{true}} \\
  \hline
  \hline
  \Code{ShowEMCsize} & size of the embedded state space, $|\mathcal{E}|$ \\
  \Code{ShowEMCdata} & all nonzero entries of the transition matrix $\P\in\Reals^{|\mathcal{E}|\times|\mathcal{E}|}$ \\
  \Code{ShowEMCprob} & $\vp[i]$ and $\sum_k\gvect{\sigma}[i, k]$ for each state $i\in\mathcal{E}$ \\
  \Code{ShowSpecial} & data about the computational effort \\
  \hline
\end{tabular}
\end{table}


% FIX AND MOVE THIS TO AN EARLIER SECTION ON "UNDERLYING STOCHASTIC PROCESSES"...
% Perhaps the best way to describe a semi-regenerative process is by beginning from
%processes that are already familiar.  The simplest stochastic process is a
%sequence of independent and identically distributed (i.i.d.) random variables
%referred to as a renewal process.
%Markov chains generalize renewal processes somewhat by allowing a limited amount
%of dependency.
%Consider now a stochastic process that starts over independently and identically
%at certain random times forming an increasing sequence $T=\{T_n : n \geq 0\}$.
%Processes that restart like this are called regenerative processes and the times $T$
%are called regeneration times.
%example with Markov chains at fixed state....
%
%So, regenerative processes generalize renewal processes,
%but regenerative processes can be generalized further just like renewal processes were
%generalized by Markov chains.
%Semi-regenerative processes generalize regenerative processes by allowing the future
%after regeneration times to depend on the state entered just after such times.
%It follows that ...
%Semi-regenerative There is an


\PRIVATE{
%        ======================
\section{Solving linear systems}
%        ======================
For many types of analysis, {\smart} must solve a linear system of the form
\[
  \vx \A = \vb
\]
for the vector $\vx$.
Since the matrix $\A$ is typically very large and
quite sparse, and is not always explicitly stored, {\smart} uses
implicit, iterative techniques \cite{Stewart1994book} for computing
$\vx$. Since the iterative methods compute $\vx$ as a fixpoint,
options analogous to the \index{converge option}\Code{converge} options are used to
adjust the computation: \Code{NumericalPrecision} defines the
desired precision $\epsilon$, \Code{NumericalPrecisionTest}
defines the \emph{stopping criterion} (either element-wise
relative or element-wise absolute precision), and
\Code{MaxNumericalIters} defines the maximum number of allowed
iterations.

\MSG{discuss \Code{RELAXATION}}

The type of iteration performed is specified using the
\index{Solver option}\Code{Solver} option,
which can be one of the following.

\begin{description}

\item{\Code{POWER} :}  The power method.
This works for systems of the form $\vp = \vp \cdot \P$, and
simply computes $\vp^{(n+1)} = \vp^{(n)} \cdot \P$ starting
with $n=0$ and continuing until subsequent vectors are within a desired
precision.

\MSG{This can be used for transient solutions of DTMCs,...}

\MSG{Does this use relaxation?}

\item{\Code{JACOBI} :}  The Jacobi iteration.
Newly computed elements of $\vx$ are not used until the next iteration.
As a result, an auxiliary vector $\vx'$ is required to store the new
values.
{\smart} is able to use the Jacobi iteration when the matrix $\A$ is stored
in a sparse format, either by rows or by columns.
{\bf This is the default.}
\MSG{Should it be GAUSS\_SEIDEL instead?}

\PRIVATE{
\item{\Code{BLOCK\_JACOBI} :}  An experimental block algorithm.
  This should not be used yet, but eventually will be used with distributed
  solutions.
}

\item{\Code{GAUSS\_SEIDEL} :}  The Gauss-Seidel iteration.
This is similar to Jacobi, except the newly computed elements of $\vx$
are used right away.
As a result, the auxiliary vector required by Jacobi is not needed with
Gauss-Seidel.
However, this requires the computation of one element of $\vx$ at a time;
as a result, the Gauss-Seidel iteration requires access to the columns of
matrix $\A$.
Therefore, Gauss-Seidel can only be used with data structures for $\A$
that allow efficient access to the matrix columns.

\MSG{Discuss relative number of iterations.}

\PRIVATE{
\item{\Code{BLOCK\_GAUSS} :}  An experimental block algorithm, just as
experimental and not working as block Jacobi.
}

\item{\Code{SOR} :} A Gauss-Seidel iteration which adaptively adjusts the
relaxation parameter.
\cite{1993IMA-SRNs}
The initial relaxation parameter is taken from the option
\index{RELAXATION option}\Code{RELAXATION}.

\PRIVATE{
\item{\Code{BICGSTAB} :} Biconjugate Gradient Stabilizer.
Also experimental (i.e., not thoroughly tested and debugged yet).
}

\end{description}

%        ===========================
\section{Representing large matrices}
%        ===========================
Since the matrices are large, blah blah blah.

{\smart} can use a variety of methods for representing the transition matrix of a Markov chain.
The methods, described below, are selected based on the option
\index{MarkovStorage option}\Code{MarkovStorage}.

%           ==============
\subsection{Sparse storage}
%           ==============
Mention rows vs columns.
Used for high-level formalisms and manipulation of phase-type vars.
For high level models, we do a memory-efficient, two-pass approach:
first count the number of entries in each row or column, so that we can
allocate an array of entries instead of using linked lists.
The second pass fills the array.
This is the default.

%           ==========
\subsection{On the fly}
%           ==========
Only high-level formalisms with expo or immediate transitions.
Only by rows.
\cite{Deavours1997onthefly}

%           =========================
\subsection{Kronecker representations}
%           =========================
Only high-level formalisms with expo or \emph{local} immediate transitions.
Mention rows vs columns.
Options to mention: \Code{MarkovStorage}, \Code{UseHoldingArray},
\Code{MatrixByRows}

The consistent requirement.

%           ===============
\subsection{Matrix diagrams}
%           ===============
Only high-level formalisms.
Mention rows vs columns.

Kronecker \cite{1999PNPM-MatrixDiagrams, MinerThesis2000} vs. non-Kronecker
\cite{Miner2001canonicalMDs}

Options to mention: \Code{MarkovStorage}, \Code{UseHoldingArray},
\Code{MergeLevel}

}
