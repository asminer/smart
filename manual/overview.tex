%
%

\section*{{\smart} overview and organization of this User Manual}

The {\smart} ({\smartmeaning}) tool is a software package to study complex
discrete-state systems. With {\smart}, it is possible to study the
logical behavior of the system, by generating the state space
underlying a complex model and asking temporal logic queries about
its dynamic behavior, as done in traditional model-checking.
Once one is satisfied that the logical behavior of the
system is both correct and correctly captured by the model,
performance, reliability, availability, and performability
measures about the system can be computed with a large variety of
state-of-the-art techniques.
{\smart} implements exact numerical solution algorithms,
approximate numerical solution algorithms,
and discrete-event simulation techniques.
These can be used in cooperation for the study of a complex
system, since multiple interacting models are supported in a
variety of ways, from a simple hierarchical solution of submodels
to a built-in fixpoint iteration mechanism which may be used to
perform sophisticated studies.
Numerical solution
algorithms are provided not only for the widely-used class of
models having an underlying continuous-time \index{Markov chain}Markov chain,
but also for the case where the underlying
process is a discrete-time Markov chain, or, under certain
conditions, even a Markov regenerative process arising from the
use of both discrete and continuous
\index{phase-type distribution}phase-type distributions in the same model.

\IGNORE{
A unifying language used to define the structure and the measures of models,
regardless of the formalism used to express them (queueing
networks, Markov chains, etc.) and of the type of study required
(performance, reliability, etc.);

Another way to deal with large and complex problems is the
use multiple processors.
This can be done in two ways
If several independent model solutions are needed, for example
to study a system under a range of different parameters,
they can be assigned to the available processors and run in parallel.
If, however, a single model is too large to be analyzed on a single processor
due to time or memory constraints, a more complex approach, using
a distributed algorithm, can be employed.
Both types of parallelism are available in {\smart}.
}

\begin{description}

\item{\bf Chapter \ref{SEC:Language}}
introduces the {\smart} input language used to define models and their
interactions.
Two types of low-level stochastic process formalisms
are presented, discrete-time and continuous-time
\index{Markov chain}Markov chains. High-level models expressed in
a formalism-specific way are instead presented in the following
individual chapters.

\begin{release}
Currently, the only high-level formalism supported by {\smart}
is an extended class of stochastic Petri nets.
More formalisms are planned for future versions.
\end{release}

\item{\bf Chapter \ref{SEC:SPN}}
describes how to define stochastic Petri nets models.
This is a very powerful and general formalism that can be used to model
discrete-state systems in virtually any domain.

\item{\bf Chapter \ref{SEC:SSGen}}
describes algorithms and options available to generate the state space
of a discrete-state model, regardless of the high-level formalism in which
it is expressed.
This fundamental step must occur before any model checking or
numerical solution activity.

\item{\bf Chapter \ref{SEC:ModelChecking}}
describes algorithms and options available to perform symbolic
model checking on a high-level model.

\item{\bf Chapter \ref{SEC:Numerical}}
describes algorithms and options available to solve the underlying
stochastic model numerically, using either traditional (explicit) or
symbolic (implicit) representations.

\begin{private}
\item{\bf Chapter \ref{SEC:Approximations}}
describes algorithms and options available to compute approximate
numerical solutions.
\end{private}

\begin{private}
\item{\bf Chapter \ref{SEC:Simulation}}
describes algorithms and options available to study models
using discrete-event simulation.
\end{private}

\item{\bf Chapter \ref{SEC:Examples}}
gives examples of {\smart} usage and reports runtimes and memory requirements
on them.

\item{\bf Appendix \ref{SEC:LanguageReference}}
lists the predefined types, functions, and options available in \smart.

\item{\bf Appendix \ref{SEC:BNF}}
formally defines the syntax of the {\smart} language using BNF notation.

\item{\bf Appendix \ref{SEC:PhaseType}}
gives key definitions and properties of
discrete and continuous phase-type distributions.

\item{\bf Release Notes}
at the end of this manual list information that a user of the current
release should know.
Other release-dependent information appears throughout this manual
\RELEASE{in blue text}.

\begin{private}
\item{\bf Internal Notes}
are designated throughout the manual for \smart{} developers or for 
work in progress.
This material is in red text and is omitted from the ``public'' user manual.
\end{private}

\end{description}

\vfill

\section*{\centering Acknowledgments}

The development of the techniques implemented in {\smart} has been partially
supported by the
National Aeronautics and Space Administration under grants
NAG-1-2168 and NAG-1-02095;
by the National Science Foundation under grants 
CSR-0546041, % Andy's CAREER
CCR-0219745, 
and
ACI-0203971;
by a joint STTR project with Genoa Software Systems,
Inc., for the Army Research Office;
and by the matching grant FED-95-011 from the Virginia Center for
Innovative Technology.
We are grateful to these organizations for making our research possible.

Any opinions, findings, and conclusions or recommendations expressed in this
material are those of the author(s) and do not necessarily reflect the views
of the National Science Foundation.

