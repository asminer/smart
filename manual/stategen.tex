%
% $Id$
%

\chapter{State space generation algorithms} \label{SEC:SSGen}

\section{Finite state spaces} \label{SEC:finite-state-spaces}

A discrete-state model expressed in a high-level formalism implicitly specifies:
\begin{itemize}
\item $\widehat{\sset}$, the set of \emph{potential states} describing
the ``type'' of states.
\item $\initstate \in \widehat{\sset}$, the \emph{initial state}
(we use boldface lower-case roman characters to denote states).
\item $\nset : \widehat{\sset} \rightarrow 2^{\widehat{\sset}}$, the
\emph{next--state function}, describing which states can be reached from
a given state in a single step.
In the formalisms we consider, $\nset$ is really expressed as a union
$\nset = \bigcup_{e \in \eset} \nset_{e}$, where~$\eset$ is a finite set
of \emph{events} and~$\nset_{e}$ is the next-state function associated
with event~$e$.
Thus, $\nset_{e}(\vi)$ is the set of states the system can enter when event $e$
occurs, or \emph{fires}, in state $\vi$.
In particular, $e$ is \emph{disabled} in $\vi$ if $\nset_e(\vi) = \emptyset$.
In the case of a Petri net with exponential timing, the set $\nset_{e}(\vi)$
can only be either empty or a singleton, since the firing of
a given transition $e$ in a given marking $\vi$
deterministically leads to a unique marking $\vj$.
However, nondeterminism can be achieved, i.e., $\nset_{e}(\vi)$ can contain
multiple elements, by using immediate transitions in the model.
\end{itemize} 

From such a high-level model, we can build its
\emph{reachable state space}, or \emph{reachability set},
$\sset \subseteq \widehat{\sset}$, as
the smallest set containing the initial state $\initstate$ and
closed with respect to~$\nset$:
$$
   \sset = \{\initstate\} \cup \nset(\initstate) \cup
   \nset(\nset(\initstate)) \cup \cdots = \nset^{\ast}(\initstate).
$$
The \emph{reachability graph} is instead the directed graph having $\sset$
as nodes, and an arc from $\vi$ to $\vj$ iff $\vj \in \nset(\vi)$.
If we are interested in the event identities, we label the arc
with the event(s) $e$ for which $\vj \in \nset(\vi)$.

When discussing discrete-state systems, it is natural to consider
a state as a collection of \emph{substates}, thus
the potential state space $\widehat{\sset}$ is the
cross-product of some fixed number $K$ of \emph{local state spaces},
$$
  \widehat{\sset} = \sset_K \times \sset_{K-1} \times \cdots \times \sset_1 .
$$

Unlike many other tools, {\smart} does not require the user to
\emph{explicitly} define the exact composition of these local state spaces.
For example, a marking $\vi$ of a Petri net is simply a vector of $K$
natural numbers, where $K$ is the number of places in the net,
but {\smart} simply needs to know this number $K$, i.e., it only needs to know
that $\vi \in \Naturals^K$, not what the possible number of tokens
in each place can be.

This is more user-friendly than requiring the
specification of redundant information (after all, the Petri net itself
implicitly defines the elements of the sets $\sset_k$ for $K \geq k \geq 1$),
but it also carries with it a responsibility:
it is up to the user to make sure that the state space is finite,
since $\widehat{\sset}$ is not anymore the cross-product of
$K$ finite (explicitly enumerated by the user) sets.
Of course, if desired, the user can still explicitly bound the local
state spaces, without having to list them.
For example, in a Petri net, this can be achieved by introducing inhibitor
arcs or complementary places that limit the number of tokens
in the places;
however, this is often not needed, as many interesting models
are naturally finite, as they are covered by invariants.

State space exploration algorithms are available in {\smart} as
functions that can be used to build measures.
We use the \Code{pn} formalism for our examples, but
the predefined functions we describe are applicable to any other
formalism in {\smart}.
The applicable algorithms are independent of the data structure used to
represent the reachability set $\sset$, although the choice for the
latter can greatly affect their memory and runtime requirements.
The available functions are:

\begin{itemize} 

\item
\Code{bigint num\_states()}\\
Builds $\sset$ if necessary, and returns the size of $\sset$.
Whenever $\sset$ is built,
if \Code{procgen} is set for options \Code{Debug} and \Code{Report},
then debugging and performance information (respectively)
for the state space generation engine are displayed.

\item
\Code{void show\_states()}\\
Builds $\sset$ if necessary, and explicitly lists the elements of $\sset$.
Option \Code{MaxStateDisplay} specifies the maximum number of states
to be displayed,
and option \Code{StateDisplayOrder} dictates the order in which states
are displayed.



\item
\Code{bigint num\_arcs()}\\
Builds the reachability graph if necessary,
and returns the number of arcs in the graph.
{\smart} builds a state-to-state transition matrix to evaluate this function,
hence duplicate arcs that lead from one state to the same state
due to different events are not counted multiple times.
Whenever the reachability graph is built,
if \Code{procgen} is set for options \Code{Debug} and \Code{Report},
then debugging and performance information (respectively)
for the reachability graph generation engine are displayed.

\item
\Code{void show\_arcs()}\\
Builds the reachability graph if necessary,
and displays it.
Option \Code{MaxArcDisplay} specifies the maximum number of arcs
to be displayed,
while options \Code{DisplayGraphNodeNames}
and \Code{GraphDisplayStyle} control the style of the graph output.
\end{itemize}

The following is a {\smart} program that computes the number of reachable states
for the classical model of the dining philosophers, where the number \Code{N}
of philosophers is an input parameter:
%
\lstinputlisting[firstline=3]{examples/phils_states.sm}
%


\section{Algorithms and data structures}

When the generation of the state space is required,
{\smart} uses the value of the \Code{ProcessGeneration} option
to determine the family of algorithms to choose from:
\begin{itemize}
  \item \Code{EXPLICIT}:\\
  States are stored explicitly, using {\smart}'s specialized state storage
  schemes.
  Explicit algorithms perform a breadth-first search where
  each state is stored and explored individually: for each
  newly-discovered state, every event is tested to see whether it is enabled
  and, if so, it is fired to obtain new states
  \cite{1997Tools-Storage}.
  The option \Code{ExplicitStateStorage} specifies which dictionary 
  data structure to use for determining if a discovered state
  is already in $\sset$ or not.

  \item \Code{MEDDLY}:\\
  States are stored symbolically, using Meddly to support
  the decision diagram representation.
  \RELEASE{Currently,
  this requires the user to specify a \emph{partition} for the model,
  which dictates how state variables are collected and mapped
  to decision diagram variables.
  }
  The option \Code{MeddlyProcessGeneration} specifies
  the generation algorithm (including explicit)
  to use with Meddly.
  \RELEASE{Some of these algorithms are not implemented yet,
  while others have prototype implementations.}
  Several other options exist to set parameters of Meddly.
\end{itemize}

