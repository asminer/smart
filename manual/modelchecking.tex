%
%

\chapter{Model checking algorithms} \label{SEC:ModelChecking}

\section{Temporal logic and model checking}

Temporal logic is concerned with properties of discrete-state models and
their evolution in time. It requires that the model define a set of states
$\sset$, a binary relation $\goesto{~}$ on states (the temporal order),
and a function that evaluates the truth value of any atomic proposition in
each state. Starting from a set of atomic predicates, it is then possible
to express more complex assertions. Let $\goesto{*}$ be the transitive and
reflexive closure of $\goesto{~}$.
Given a predicate~$p$ and a state $\vi$,
we can discuss basic questions of the type :

\begin{itemize}
\item
(in the future) $\opF{}p$ holds in~$\vi$ if and only if there exists a state $\vj$
such that $\vi \goesto{*} \vj$ and $p$ holds in~$\vj$.
\item
(in the past) $\opP{}p$ holds in~$\vi$ if and only if there exists a state $\vj$
such that $\vj \goesto{*} \vi$ and $p$ holds in~$\vj$.
\item
(globally) $\opG{}p$ holds in~$\vi $ if and only if~$p$ always holds in
the future states, equivalent to~$\neg\opF{}\neg p$.
\item
(historically) $\opH{}p$ holds in $\vi$ if and only if~$p$ has always held in
the past prior to $\vi$, equivalent to~$\neg\opP{}\neg p$.
\end{itemize}

The adjective ``temporal'' simply refers to the succession of events
leading from a state to another, not to the actual time of their
occurrence. In the literature, both \emph{linear} time logic (LTL) and
\emph{branching} time logic (CTL) have received much attention
\cite{Clarke1999book,McMillan1992thesis};
{\smart} implements the latter. In CTL, operators
occur only in pairs.  The first operator, the path quantifier, is
either~$A$ (on all paths), or~$E$ (there exists a path), while the second
one, the tense operator, is chosen from among~$X$ (next),~$F$ (future, or
finally),~$G$ (globally, or generally), and~$U$ (until). 
Fig.~\ref{FIG:CTLops} illustrates the meaning of each of the eight operator
combinations in CTL.
Note that {\smart} also allows the possiblity to talk about the past:
$\opY$ is time-reversed $\opX$,
$\opP$ is time-reversed $\opF$,
$\opH$ is time-reversed $\opG$,
and 
$\opS$ is time-reversed $\opH$.


\begin{figure}
  \centering
  \includegraphics[scale=0.8]{figures/CTL.pdf}
  \caption{Examples of CTL formulae. The root node satisfies the
  corresponding formula.}
  \label{FIG:CTLops}
\end{figure}

Given a discrete-state model, a CTL formula uniquely identifies a set of
states that satisfy it. Thus the model checking activity can be thought of
as the identification of the set of states satisfying the formula.
{\smart} uses the type \Code{stateset} to refer to a set of states
of a particular model, e.g., those states satisfying a CTL state formula.
{\smart} supports both \emph{explicit} and \emph{symbolic} model checking.
As discussed in Sec.~\ref{SEC:SSGen},
the state space $\sset$ can be generated and stored explicitly or symbolically;
if $\sset$ is stored explicitly, then all \Code{stateset}s for that model
will also be stored explicitly,
and explicit model checking algorithms will be used
(options allow selection between different algorithms).
Otherwise, if $\sset$ is stored as a decision diagram using Meddly,
then so will all \Code{stateset}s for that model,
and symbolic model checking algorithms will be used.



\section{Model checking in {\smart}}

The set of atomic propositions in {\smart} is represented by boolean
expressions inductively constructed over a set of terms. For Petri net
models, the set of terms includes objects of type \Code{proc bool} and
\Code{proc int}, plus the formalism-dependent types \Code{place} and
\Code{trans}. These terms can be combined using boolean operators
(negation, conjuction, disjunction) between boolean subterms, relational
(equal, less than, etc.) and arithmetic (addition, multiplication, etc.)
operators between arithmetic subterms, and the following
formalism-dependent functions: 
\begin{itemize} 
  \item \Code{proc int tk(place p)}, which returns the number of tokens in place \Code{p}. 
  \item \Code{proc bool enabled(trans t)}, which returns \Code{true} if transition
        \Code{t} is enabled, \Code{false} otherwise. 
\end{itemize}

{\smart} provides the type \Code{stateset} to define sets of states in a model. 
Any set of states defined for a model is a subset of the ``universe'', 
which for explicit storage is the set of reachable states $\sset$,
while for symbolic storage is the set of potential states $\potsset = \sset_K \times \cdots \times \sset_1$
(see Sec.~\ref{SEC:SSGen}).
Hence, it is possible to define sets containing both reachable and unreachable states. 
If desired, the
unreachable states can be eliminated from a set by intersecting it with
the state space $\sset$, which is specified in {\smart} as
\Code{reachable}.
\TBD{Do we want to make these consistent, i.e., make a stateset always be a subset
of the reachable states?}

Inside a model block, 
a \Code{stateset} for an atomic proposition may be constructed using function
\begin{lstlisting}
    stateset potential(proc bool p)
\end{lstlisting}
where $p$ is a boolean expression that depends on the model state,
and the resulting \Code{stateset} contains the states in which $p$
evaluates to true.
The eight CTL pairs shown in Fig.~\ref{FIG:CTLops},
and their time-reversed counterparts,
correspond to functions in {\smart} with the same name,
that must also be called within a model block.
For example,
\begin{lstlisting}
    stateset EU(stateset p, stateset q)
\end{lstlisting}
returns the \Code{stateset} containing states
satisfying $\opE\,p\,\opU\,q$.
Additionally,
operators \Code{!} (negation / complement), \Code{\&} (conjunction / intersection), 
\Code{|} (disjunction / union), \verb|\| (set difference), and \Code{->} (implication)
may be applied to \Code{stateset}s,
and are not required to be used within a model block,
although
it is an error to use \Code{stateset}s from different models
(in fact, from different state spaces)
as operands to the same operator.
The cardinality of a \Code{stateset} may be obtained with
\begin{lstlisting}
    bigint card(stateset p)
\end{lstlisting}
which, in the case of symbolic model checking,
may return a \emph{huge} value.
If instead we only care if a set is empty or not,
it is much more efficient to use
\begin{lstlisting}
    bool empty(stateset p)
\end{lstlisting}
which is (mathematically, but not computationally)
equivalent to $0==card(p)$.
Finally,
the elements of a \Code{stateset} may be printed
using the \Code{print} function;
option \Code{StatesetPrintIndexes}
dictates whether the set is displayed as a list of state indexes
or a list of states.


\begin{private}
\section{Counterexamples and witnesses}

\TBD{NOT IMPLEMENTED YET; UPDATE AFTER THAT}

One of the most attractive features of model checking is its ability to provide
counter-examples and witnesses for the properties that are checked. The
two notions are dual. Whenever a formula prefixed with the universal path
quantifier,~$A$, is not verified, the model checker provides the user with
an execution trace that proves the negation of the formula (a
\emph{counter-example}). Similarly, for a valid formula prefixed with the
existential path quantifier,~$E$, the model checker delivers a
\emph{witness} to that.

In both cases, the execution traces are in fact proofs for
existentially quantified formulae: $EX p$, $EF p$, $E[p~U~q]$, or $EG p$,
where $p$ is an arbitrary CTL formula.
Among these, only the last three cases are interesting,
since $EX p$ is trivially computed in one step.
$EF p$ and $E[q~U~p]$ witnesses are single finite paths that ``show'' how the
target set of states (satisfying formula $p$ in this case) can be reached
from the initial set of states through a sequence of valid transitions.
In addition, for the $EU$ witness, $q$ must hold along the entire path until
the last state, where $p$ holds.
Finally, an $EG p$ witness is a strongly connected component (SCC) in the
reachability graph where formula $p$ holds on all states.
A typical (non-trivial) SCC is a cycle, with or without chords.
In our model checker, the computation of $EG$ traces is reduced to the
equivalent problem of finding and concatenating two $EU$ traces
forming a cycle plus possibly a third trace leading to that cycle.
Thus, we can limit our discussion to the traces for $EF$ and $EU$.

Since a trace is usually meant to be examined by a human, it is desirable
to compute a minimal-length one.
However, finding such a trace for an
arbitrary formula is an NP-complete problem, thus a sub-optimal trace is
sought in most cases, usually the one that is found first. Due to the
heavy computation price to be paid, our model checker does not compute the
execution traces automatically for each CTL formula that is verified, but
only upon request, by calling one of the three functions:

\begin{itemize}
\item 
\LIGHTGREY{
\Code{bool EFtrace(stateset p, stateset q)}\\
  Finds and prints a witness for $EFq$, of shortest length, starting from
  \Code{p}.
  Returns \Code{true} if such a trace exists, \Code{false} otherwise. 
}
\item 
\LIGHTGREY{
\Code{bool EUtrace(stateset p, stateset q, stateset r)}\\
  Finds and prints a minimal length witness for $E[q~ U~ r]$ that starts at
  \Code{p}.
  Returns \Code{true} if such a trace exists, \Code{false} otherwise. 
}
\item \Code{mdd EGtrace(mdd p)}\\
  Finds and prints a shortest witness for $EGq$, starting from the initial state.
  %Returns \Code{true} is such a strace exists, \Code{false} otherwise. 
  \PRIVATE{{\smart} computes this trace by concatenating two traces.}
\end{itemize}

\LIGHTGREY{
The output of the above functions is the sequence of states in the
execution trace along with the identity of the event fired at each step.
See Fig.~\ref{FIG:TraceOutput} for details on the output format. 
In addition to these three functions, we provide a distance function:
}
\begin{itemize}
\item 
\LIGHTGREY{
\Code{int} \Code{dist(stateset p, stateset q)} \\
  Returns the length of a shortest path from any state in \Code{p}
  to any state in \Code{q}.
  If no such path exists, it returns \Code{infinity}.
}
\end{itemize}

\LIGHTGREY{
The above functions restrict the origin of the trace to a set of states
\Code{p}.
This is more general than assuming that the execution trace always starts
with the initial state.
Our format is more flexible in that it would eventually enable the
concatenation of traces, as required when verifying nested CTL formulae.
The original semantics of the traces is simply obtained by setting the argument
\Code{p} of the functions to the atomic proposition \Code{initialstate}.
}

The computation of traces in {\smart} is implemented by means of symbolic
data structures.
The complete set of traces and the distance function can
be computed with a standard approach that uses a forests of MDDs (with
shared nodes, for efficiency) in conjunction with breadth-first exploration.
A second approach using edge-valued decision diagrams (a generalized form
of EVBDDs \cite{Lai1996EVBDDs}) is also available;
this is much more efficient due to its ability to apply the saturation
strategy, not just breadth-first exploration, but it can only be used to
compute the distance function \Code{dist}, or $EF$ traces.
A third approach uses Algebraic Decision Diagrams (ADDs \cite{Bahar1997ADDs}),
and it, also, can use either a saturation or a breadth-first approach.
The option \Code{TraceStorage}, with possible
values \Code{SHMDD\_BFS}, \Code{ADD\_SAT}, \Code{ADD\_BFS},
\Code{EVMDD\_BFS}, \Code{EVMDD\_SAT} (default) sets the data structure and
iteration strategy used for these two functions
(see \cite{2002FMCAD-EVMDD} for a more detailed description).

Moreover, when shared MDDs are used, the type of BFS exploration is
governed by the option \Code{BFSTrace}, with possible values
\Code{FORWARD}, \Code{BACKWARD}, and \Code{JOIN} (default).
For the first two values, the behavior is symmetric.
A \Code{FORWARD} ($EF$ or $EU$) trace starts at the source and reaches the
target by using a forward BFS exploration, which therefore yields the
shortest length for the generated trace.
A \Code{BACKWARD} trace starts at the target and backtracks to the source.
The \Code{JOIN} strategy uses a divide-and-conquer handshaking algorithm
which starts the BFS exploration at both ends and meets in the middle.
The trace generation continues recursively on the two segments
delimited by the junction point.

We give here a very simplistic guideline to the complexity analysis of the
three variants of trace construction. If~$d$ is the distance from the
source to the target set of states (the length of the shortest path
between them), then \Code{FORWARD} and \Code{BACKWARD} algorithms store
$O(d)$ intermediate sets of states, and take $O(d)$ steps to complete,
while the \Code{JOIN} algorithm stores only two sets of states at all
times, but takes $O(d\log d)$ steps to compute the trace.
The user should adjust the settings according to the desired time/space
tradeoffs.
Note that symbolic data structures may be
counter-intuitive when it comes to representing sets, as the necessary
space is determined by the number of MDD nodes, and not the number of
encoded states (it is often the case that smaller sets take much more
memory than larger sets).
\end{private}

\section{Examples}

\begin{figure}
  \lstinputlisting[firstline=3]{examples/phils_dead.sm}
\caption{Printing the deadlocked states in the dining philosophers model.}
\label{FIG:PhilsDeadlock}
\end{figure}


The code in Fig.~\ref{FIG:PhilsDeadlock} generates and prints
the deadlocked states (if any exist) in the dining philosophers model. 
The \Code{stateset} \Code{NotAbsorb} contains all potential states that have a
some successor, thus all non absorbing states. The \Code{stateset}
\Code{Deadlocked} contains exactly the reachable states not in \Code{NotAbsorb}, 
that is, all reachable states with no successor.
Outside the model,
we check if set \Code{Deadlocked} is empty, and if not, 
we print the deadlocked states.
Note we could have simply printed the set;
{\smart} will display an empty set if asked to print an empty \Code{stateset}.

\begin{figure}
  \centering
  \includegraphics[scale=0.64]{figures/kripke1.pdf}
  \lstinputlisting[firstline=3]{examples/kripke1.sm}
\caption{Illustrating temporal operators on a small example.}
  \label{FIG:kripke1}
\end{figure}

The code in Fig.~\ref{FIG:kripke1} illustrates the difference between
various temporal logic operators.
The model is partitioned into two submodels, each containing a cycle.
The inhibitor arcs in the {\smart} input file are added to ensure the correct
size of the local spaces, they are needed only for methods that pre-generate
the local state spaces in isolation.

\begin{itemize}
\item
The potential state space is $\sset_2 \times \sset_1$,
with $\sset_2 = \{a, b, c, \epsilon\}$ and $\sset_1 = \{\epsilon, d, e\}$,
thus it contains $12$ states
(by  $\epsilon$ we mean the local marking where all places are empty,
by ``$a$'' we mean the local marking where place $a$ contains one token
and places $b$ and $c$ are empty, and so on).
\item
The state space contains five states, corresponding to the token being in
exactly one of the five places.  
\item
Three potential states have a token in $a$,
$\{(a,\epsilon),(a,d),(a,e)\}$, but only one, $(a,\epsilon)$, is reachable.
\item
The reachable states that can reach a state with a token in $a$ are
$\{(a,\epsilon), (b,\epsilon), (c,\epsilon)\}$.
If we included the non-reachable states, this set would
contain nine states (same reasoning as in the previous item).
\item
If we search for states that have an outgoing infinite path (i.e., states
included in a strongly connected component) where $a$ always contains a token,
the set $\{(a,d), (a,e)\}$ is found, but these states are non-reachable,
so they are not printed when we show the set \Code{ex.rEGa}.

\item
Instead, if we search for a reachable strongly connected component of states
where $a$, or $b$, or $c$ always contain a token,
we again find nine potential states or three actual states.
\item
The $\opE[c~U~d]$ query finds states $(c,\epsilon)$ and $(d,\epsilon)$,
while $\opA[c~U~d]$ finds only state $(d,\epsilon)$.


\item
The query $\opE[(a \vee b \vee c)~\opU~d]$ 
finds the set of four states
$\{(a,\epsilon), (b,\epsilon), (c,\epsilon), (d,\epsilon)\}$,
while the query $\opA[(a \vee b \vee c)~\opU~d]$
finds only state $(d, \epsilon)$;
this is because it is possible for the token to circulate
among places $a$, $b$, and $c$ indefinitely.
If we instead use a \emph{stochastic} Petri net model,
where transitions have firing rates,
then the underlying structure becomes a continuous-time Markov chain
(rather than a kripke structure),
and since any path where the token remains indefinitely in places
$a$, $b$, and $c$ has probability measure zero,
in this case the query $\opA[(a \vee b \vee c)~\opU~d]$
will return the same four states as
$\opE[(a \vee b \vee c)~\opU~d]$.

\end{itemize}




\begin{private}
\TBD{Need to implement execution traces!}


\begin{figure}
\begin{lstlisting}
spn k(int n) := {
  place pm1, pback1, pkan1, pout1,    pm2, pback2, pkan2, pout2,
        pm3, pback3, pkan3, pout3,    pm4, pback4, pkan4, pout4;
  trans tin1, tredo1, tok1, tback1,   tin2, tredo2, tok2, tback2, tout2,
        tredo3, tok3, tback3,         tredo4, tok4, tback4, tout4;
  partition(pm1:pback1:pkan1:pout1,   pm2:pback2:pkan2:pout2, 
	    pm3:pback3:pkan3:pout3,   pm4:pback4:pkan4:pout4);
  firing(tin1:expo(1.0), tredo1:expo(0.36), tok1:expo(0.84), 
	 tback1:expo(0.3), tin2: expo(0.4), tredo2:expo(0.42), 
	 tok2: expo(0.98), tback2:expo(0.3), tout2: expo(0.5), 
	 tredo3:expo(0.39), tok3:expo(0.91), tback3:expo(0.3), 
	 tredo4:expo(0.33), tok4:expo(0.77), tback4:expo(0.3), tout4:expo(0.9)
	); 
  arcs(pkan1:tin1, tin1:pm1, pm1:tredo1, pm1:tok1, tredo1:pback1,
       tok1:pout1, pback1:tback1, tback1:pm1, pout1:tin2, tin2:pkan1,
       pkan2:tin2, tin2:pm2, pm2:tredo2, pm2:tok2, tredo2:pback2,
       tok2:pout2, pback2:tback2, tback2:pm2, pout2:tout2, tout2:pkan2,
       pkan3:tin2, tin2:pm3, pm3:tredo3, pm3:tok3, tredo3:pback3,
       tok3:pout3, pback3:tback3, tback3:pm3, pout3:tout2, tout2:pkan3,
       pkan4:tout2, tout2:pm4, pm4:tredo4, pm4:tok4, tredo4:pback4,
       tok4:pout4, pback4:tback4, tback4:pm4, pout4:tout4, tout4:pkan4);
  init(pkan1:n, pkan2:n, pkan3:n, pkan4:n); 
  stateset F := intersection(reachable,
           potential(tk(pout4)==n & tk(pout1)==n & tk(pout2)==n & tk(pout3)==n));
  int  D := dist(initialstate, F);
  bool T := EFtrace(initialstate, F);
};
# StateStorage  MDD_SATURATION
# BFSTrace FORWARD
int N := read_int("N");
print("Distance to the farthest state: ", k(N).D, ".\n");
print("Execution trace:"); compute(k(N).T);
\end{lstlisting}
\caption{Generating CTL execution traces in {\smart}.}
\label{FIG:KanbanTraces}
\end{figure}


The code in Fig.~\ref{FIG:KanbanTraces} illustrates the use of execution
traces on a kanban manufacturing system example (see Fig.~\ref{FIG:kanban}
for the graphical model).
In this case, the computation of the witness is influenced by the value of the
\Code{BFSTrace} option.
Figure \ref{FIG:TraceOutput} shows the two different shortest-length witnesses
obtained when this option is set to either \Code{FORWARD} or \Code{BACKWARD}.

\begin{figure}
When option \Code{BFSTrace} is set to \Code{FORWARD}:
\begin{lstlisting}
EF Trace of length 14
* Starting from state 	=>  { pkan1:1 pkan2:1 pkan3:1 pkan4:1 }
* Step 1: fire event tin1	=>  { pm1:1 pkan2:1 pkan3:1 pkan4:1 }
* Step 2: fire event tok1	=>  { pout1:1 pkan2:1 pkan3:1 pkan4:1 }
* Step 3: fire event tin2	=>  { pkan1:1 pm2:1 pm3:1 pkan4:1 }
* Step 4: fire event tin1	=>  { pm1:1 pm2:1 pm3:1 pkan4:1 }
* Step 5: fire event tok1	=>  { pout1:1 pm2:1 pm3:1 pkan4:1 }
* Step 6: fire event tok2	=>  { pout1:1 pout2:1 pm3:1 pkan4:1 }
* Step 7: fire event tok3	=>  { pout1:1 pout2:1 pout3:1 pkan4:1 }
* Step 8: fire event tout2	=>  { pout1:1 pkan2:1 pkan3:1 pm4:1 }
* Step 9: fire event tin2	=>  { pkan1:1 pm2:1 pm3:1 pm4:1 }
* Step 10: fire event tin1	=>  { pm1:1 pm2:1 pm3:1 pm4:1 }
* Step 11: fire event tok1	=>  { pout1:1 pm2:1 pm3:1 pm4:1 }
* Step 12: fire event tok2	=>  { pout1:1 pout2:1 pm3:1 pm4:1 }
* Step 13: fire event tok3	=>  { pout1:1 pout2:1 pout3:1 pm4:1 }
* Step 14: fire event tok4	=>  { pout1:1 pout2:1 pout3:1 pout4:1 }
\end{lstlisting}

When option \Code{BFSTrace} is set to \Code{BACKWARD}:
\begin{lstlisting}
EF Trace of length 14
* Starting from state 	=>  { pkan1:1 pkan2:1 pkan3:1 pkan4:1 }
* Step 1: fire event tin1	=>  { pm1:1 pkan2:1 pkan3:1 pkan4:1 }
* Step 2: fire event tok1	=>  { pout1:1 pkan2:1 pkan3:1 pkan4:1 }
* Step 3: fire event tin2	=>  { pkan1:1 pm2:1 pm3:1 pkan4:1 }
* Step 4: fire event tok3	=>  { pkan1:1 pm2:1 pout3:1 pkan4:1 }
* Step 5: fire event tok2	=>  { pkan1:1 pout2:1 pout3:1 pkan4:1 }
* Step 6: fire event tout2	=>  { pkan1:1 pkan2:1 pkan3:1 pm4:1 }
* Step 7: fire event tok4	=>  { pkan1:1 pkan2:1 pkan3:1 pout4:1 }
* Step 8: fire event tin1	=>  { pm1:1 pkan2:1 pkan3:1 pout4:1 }
* Step 9: fire event tok1	=>  { pout1:1 pkan2:1 pkan3:1 pout4:1 }
* Step 10: fire event tin2	=>  { pkan1:1 pm2:1 pm3:1 pout4:1 }
* Step 11: fire event tok3	=>  { pkan1:1 pm2:1 pout3:1 pout4:1 }
* Step 12: fire event tok2	=>  { pkan1:1 pout2:1 pout3:1 pout4:1 }
* Step 13: fire event tin1	=>  { pm1:1 pout2:1 pout3:1 pout4:1 }
* Step 14: fire event tok1	=>  { pout1:1 pout2:1 pout3:1 pout4:1 }
\end{lstlisting}
\caption{The output of $EF$ witnesses, depending on the setting of the
\Code{BFSTrace} option.}
\label{FIG:TraceOutput}
\end{figure}

\end{private}
