%
%

\chapter{Stochastic Petri nets}  \label{SEC:SPN}

\TBD{
  Rewrite this chapter, make it ``Petri nets'',
  cover both logical and stochastic aspects.
}

The \index{stochastic Petri net}stochastic Petri net
formalism is based on a stochastic extension of
(untimed) Petri nets \cite{Murata1989,Peterson1981,Petri1962,Reisig1985}.
The type used to declare a stochastic Petri net in {\smart} is
\index{pn}\Code{pn}.
Depending on the timing associated with the transitions of the
net and on the way they can interrupt each other,
the underlying model can be a discrete-time Markov chain, a continuous-time
Markov chain, a \index{semi-regenerative process}semi-regenerative process,
or a generalized semi-Markov process
\cite{Ajmone1984,Chiola1993b,1995StewartBook-DiscreteSPN,1993IMA-SRNs,
1994IEEETSE-SPNcharacterization,Molloy1981,Natkin1980}.
{\smart} automatically determines the type of underlying stochastic process,
and uses this information to decide the solution engines that can be
applied to it.
The conditions for this determination are as follows (see
Appendix \ref{SEC:PhaseType} for a discussion of DDP and DCP distributions):
\begin{itemize}
\item
If all the transitions in the net have a DDP distribution
(including, possibly, the degenerate case of a constant equal to zero),
the underlying process is determined to be a DTMC (discrete-time Markov chain).
\item
If all the transitions in the net have a DCP distribution,
(including, possibly, the degenerate case of a constant equal to zero),
the underlying
process is determined to be a CTMC (continuous-time Markov chain).
\item
If all the transitions in the net have a DDP or DCP distribution
(including, possibly, the degenerate case of a constant equal to zero),
and certain conditions on the concurrent enabling of the transitions
with DDP distributions are satisfied (essentially, guaranteeing that their
possible firing times are always synchronized modulo the time step
\cite{Jones-NASA-00,2001PNPM-PDPN}), the underlying process is determined
to be a SRP (\index{semi-regenerative process}semi-regenerative process), also
called MRGP (\index{Markov-regenerative process}Markov-regenerative process).
\item
In all the other cases, the process is asssumed to be a GSMP
(generalized semi-Markov process).
\end{itemize}

Note that some aspects of the timing and stochastic behavior defined for the
model do affect its logical analysis described in Chapters \ref{SEC:SSGen}
and \ref{SEC:ModelChecking}.
In particular, \emph{immediate} events (transitions of the Petri net
with a constant zero firing time distribution)
are treated in a fundamentally different way from ordinary
\emph{timed} events: the states in which they are enabled
have sojourn times equal zero, and are not considered part of
the reachable state space (\emph{tangible reachability set} in
GSPN parlance \cite{Ajmone1995book}).



\section{Defining a stochastic Petri net model in {\smart}} \label{SEC:SPNdef}

An example of a \index{pn}{stochastic Petri net} definition is:
%
\lstinputlisting[firstline=3]{examples/pn1a.sm}
%
\begin{figure}
  \centering
  \includegraphics[scale=0.75]{figures/cs1.pdf}
  ~~~~~
  \includegraphics[scale=1]{figures/cs1CTMC.pdf}
  %
  \caption{A simple Petri net (left) and its underlying CTMC (right).}
  \label{FIG:cs1}
\end{figure} 
%
The code defines the stochastic Petri net \Code{cs1} illustrated
in Fig.~\ref{FIG:cs1} on the left.
Since all firing times are exponentially distributed, the underlying
process is a CTMC.
This CTMC is shown in Fig.~\ref{FIG:cs1} on the right, where the state
structure is given by the pair
``(number of tokens in \Code{p}, number of tokens in \Code{q})''.
This \Code{pn} model also defines a measure \Code{m1}, the
expected number of tokens in place \Code{p} in steady state.
This measure can be requested in output as usual:
%
\lstinputlisting[firstline=3]{examples/pn1b.sm}
%
will cause its value to be computed and printed to the current output stream.

We observe that the declarations of \Code{p} and \Code{q}, and of \Code{t},
\Code{u}, and \Code{v} are also definitions:
just like type \Code{state} for discrete or continuous-time Markov chains,
types \Code{place} and \Code{trans} cannot be assigned a value,
they are essentially \Code{void} types.

Had the distributions for the firing times been specified as
\begin{lstlisting}
  firing(t:geom(0.2), u:geom(0.3), v:geom(0.1*tk(p)));
\end{lstlisting}
the resulting underlying process would have been a DTMC and it still would
be solvable numerically.
\begin{release}
However, currently, \smart{} can numerically analyze only
\Code{expo} and 0 firing times.
\end{release}

In the presence of mixtures of \Code{ph int} and \Code{ph real} firing time
distributions, the result might be a semi-regenerative process.
This is the case if we had specified
\begin{lstlisting}
  firing(t:geom(0.2), u:expo(0.3), v:erlang(5,0.02));
\end{lstlisting} 
In this case, it is still possible to compute a numerical solution.
% In this case, {\smart} is still able to compute a numerical solution.
\begin{release}
Again, the current version of \smart{} cannot handle this case.
\end{release}

In more complex cases, i.e., in the presence of complex mixtures of
\Code{ph int} and \Code{ph real} firing time distributions,
or of any \Code{rand int} or \Code{rand real} firing time distribution,
the resulting process would have to be studied using discrete-event simulation.
If a numerical solution is requested in this case, {\smart} will still
attempt to generate the correct state space;
as soon as it discovers a state where the required semi-regenerative behavior
is not present, it halts and gives and gives an error.
Thus, requiring a numerical solution when the process is not of the appropriate
type does not lead to incorrect results, but it can cause {\smart} to run for
a long time before issuing the correct error message.
\begin{release}
Currently, {\smart} does not support discrete-event simulation
of Petri nets.
\end{release}







\section{Marking-dependent expressions}
\label{SEC:spn-predefined-state-functions}

A \emph{current marking} is assumed when specifying the dynamic behavior
of the stochastic Petri net.
Thus, for example, \Code{tk(p)} is an integer quantity representing
the number of tokens in place \Code{p} as the net evolves over time:
we say its type is \Code{proc int}, where ``\Code{proc}'' stands
for ``process'', since the value of \Code{tk(p)} depends on the
state of a process evolving over time.

A statement such as
\begin{lstlisting}
  guard(t:enabled(u)|tk(p)>5);
\end{lstlisting}
can be used to specify a guard
for \Code{t} that disables \Code{t} when \Code{u} is not enabled and
when \Code{p} contains no more than five tokens.
Indeed, it is even possible in {\smart} to define nets having marking-dependent
arc cardinalities \cite{1994ICATPN-Vararcs}
(also known as \emph{self-modifying nets}):
\begin{lstlisting}
  arcs(p:t:tk(p));
\end{lstlisting}
specifies an arc from \Code{p} to \Code{t} with
cardinality equal to the (current) number of tokens in \Code{p};
thus, when \Code{t} fires, place \Code{p} becomes empty
(unless there is also an output arc from \Code{t} to \Code{p}).
When such expressions are evaluated in light of a \emph{current marking},
they are deterministic quantities, given that this marking is known.

The initial marking specification cannot be marking-dependent, thus
a statement such as
\begin{lstlisting}
  init(p:1, q:4*tk(p));
\end{lstlisting}
is illegal.
Also, there cannot be circular dependencies in the behavior of the
stochastic Petri net.
Thus, for example, we can add the statement
\begin{lstlisting}
  guard(t:!enabled(u));
\end{lstlisting}
to disable \Code{t} when \Code{u} is enabled, or add the statement
\begin{lstlisting}
  guard(u:!enabled(t));
\end{lstlisting}
to disable \Code{u} when \Code{t} is enabled.
However, adding both statements causes the behavior to be undefined, so
{\smart} issues an error in this case.

The functions \Code{avg\_ss}, \Code{avg\_at}, \Code{avg\_acc}, and their
``\Code{prob}'' siblings can be used to specify measures for the model.
They require a \Code{real} expression for their first argument
(or \Code{int}, of course, since it can be automatically promoted).
Arbitrarily complex expressions can be used for this argument:
in the example of Section \ref{SEC:SPNdef},
instead of just \Code{avg\_ss(tk(p))}, the expected number of tokens
in place \Code{p} in steady state, we could have asked for the
steady-state average of any marking-dependent expression.

The following lists the functions available to build marking-dependent
expressions for \Code{pn} models.
They all assume the notion of a current marking.
For a complete,
release-specific list of functions available for \Code{pn} models,
see the ``pn'' help topic (e.g., \Unix{smart -h pn}).

\begin{itemize}
\item
\begin{private}
\Code{proc~bool~empty(place~p);}\\
Returns \Code{true} if place \Code{p} is empty, \Code{false} otherwise.
\end{private}

\item
\Code{proc~bool~enabled(trans~t);}\\
Returns \Code{true} if transition \Code{t} is enabled,
\Code{false} otherwise.

\item
\Code{proc~int~tk(place p);}\\
Returns the number of tokens in place \Code{p}.

\begin{private}
\item
\Code{proc~bool~fires(trans~t);}\\
Returns \Code{true} if transition \Code{t} fires, \Code{false} otherwise.
This is an impulse-type process, and it is required to specify timing
aspects related to structural state changes in the net, such as the
\Code{restart} behavior.
\end{private}

\item
\Code{proc~real~rate(trans~t);}\\
Returns the rate of transition \Code{t}.
\begin{release}
Currently, it can only be used if \Code{t} has an \Code{expo}
firing distribution.
\end{release}

\end{itemize}



\begin{private}

\section{Formalism-specific functions for stochastic Petri net models}
\label{SEC:spn-predefined-builder-functions}

\TBD{should we delete this list?  The \smart{} internal documentation
is always more up to date than what we list here}

\TBD{List needs updating, most parameters are ``proc'' now}

The block defining a stochastic Petri net can contain any number
of calls to the following predefined functions:

\begin{itemize}

\item
\Code{void~init(place:int~p1:n1,~place:int~p2:n2,~...);}\\
Sets the initial number of tokens in place \Code{p}$i$ to the value of the
(non-marking-dependent) \Code{int} expression \Code{n}$i$.
Any place for which an initial number of tokens is not explicitly set
by this function is assumed to be initially empty.
If the expression \Code{n}$i$ for a place \Code{p}$i$ evaluates to zero,
a warning is issued (since this is already the default value).

\item
\Code{void~arcs($node$:$node$:int~a1:b1:n1,~$node$:$node$:int~a2:b2:n2,~...);}\\
Sets the cardinality of the input (if $node$:$node$ is \Code{place:trans})
or output (if  $node$:$node$ is \Code{trans:place}) arc
from \Code{a}$i$ to \Code{b}$i$ to the value of the
(possibly marking-dependent) \Code{int} expression \Code{n}$i$.
If an arc is not listed, it is assumed to have cardinality zero
(i.e., there is no arc).
If a pair $node$:$node$ appears but no cardinality is specified for it, the
corresponding arc is assumed to have cardinality one.

\item
\Code{void~inhibit(place:trans:int~a1:b1:n1,~place:trans:int~a2:b2:n2,~...);}\\
Exactly analogous to \Code{arcs}, except that it specifies the cardinality of
inhibitor arcs (which can only connect places to transitions) thus,
if an arc is not listed, it is assumed to have cardinality infinity
(i.e., there is no inhibitor arc).

\PRIVATE{
\item
\Code{void~prio(trans:trans~t1:s1,~trans:trans~t2:s2,~...);} \TBD{It's ``priority'' now}\\
Sets an acyclic preselection priority relation \cite{1995StewartBook-DiscreteSPN}
for transition \Code{t}$i$ over transition \Code{s}$i$.

\item
\Code{void~post(trans:trans~t1:s1,~trans:trans~t2:s2,~...);} \TBD{not implemented}\\
Sets an acyclic postselection priority relation \cite{1995StewartBook-DiscreteSPN}
for transition \Code{t}$i$ over transition \Code{s}$i$.
} 

\item
\Code{void~guard(trans:bool~t1:b1,~trans:bool~t2:b2,~...);}\\
Sets the guard for transition \Code{t}$i$ to the value of the
(normally marking-dependent) \Code{bool} expression \Code{b}$i$.
If a guard is not specified for a transition, it is assumed to be
the constant \Code{true}.

\PRIVATE{
\item
\Code{void~immediate(trans~t1,~trans~t2,~...);}\\
Sets the firing time distribution for the argument transitions to the constant
zero, that is, defines them as immediate transitions.
Any transition not defined to be immediate is taken to be timed,
that is, its firing time is positive with probability one.
\TBD{THIS FUNCTION IS GONE NOW}
}

\item
\Code{void~firing(trans:real~t1:f1,~trans:real~t2:f2,~...);}\\
Sets the firing time distribution for timed transition \Code{t}$i$ to the value
of the (possibly marking-dependent) expression \Code{f}$i$.
It is an error to set the firing distribution of an immediate transition.
For solution methods using any \Code{StateStorage} other than \Code{SPLAY},
each timed transition must have an exponential firing distribution,
i,e., it must be of the form \Code{expo($e$)},
where $e$ is a (possibly marking-dependent) \Code{real} expression,
which must evaluate to a positive finite value in any marking
where the transition is enabled.
When \Code{SPLAY} is used, the firing time distribution can instead be
\Code{ph int} or \Code{ph real} as well.
\PRIVATE{For simulation, any \Code{rand int} or \Code{rand real} distribution
with non-negative support can be specified.}
\PRIVATE{We need to be explicit about what can be used for \Code{f}$i$.}
\PRIVATE{The value \Code{const(infinity)}
can be specified for the distribution of the firing time.
This effectively causes the transition not to fire without being disabled.
TRUE???}

\item
\Code{weight(trans:real~t1:r1,~trans:real~t2:r2,~...);}\\
Sets the firing weight for transition \Code{t}$i$ to the value of the
(possibly marking-dependent) \Code{real} expression \Code{r}$i$,
which must evaluate to a positive and finite value in any marking
where the transition is enabled.
Internally, {\smart} normalizes weights so that they sum to one.
For example, \Code{weight(t:1,u:1,v:2)} specifies that transitions
\Code{t}, \Code{u}, and \Code{v}, have firing probability proportional
to one, one and two, respectively, when any subset of them attempts
to fire at the same time.
An error occurs if a transition has weight zero.

Each call to \Code{weight} defines a new ``weight class'',
a transition can appear at most in one class,
and weights are meaningful only within a class.
If a transition does not appear in any class, it is assumed to be in a class
by itself with a weight of one.

Weights are meaningful only within a class.  A ``stochastic
confusion'' arises if two transitions in different classes can fire at
the same time, and the choice of which one to fire first affects the
measures \cite{1996MASCOTS-StochasticConfusion}.
{\smart} issues an error whenever
it detects a stochastic confusion while generating the state space.
The model must then be corrected by modifying the enabling of the
interested transitions or their weight class specification.


\PRIVATE{
\item
\Code{void~restart(trans:bool~t1:b1,~trans:bool~t2:b2,~...);}\\
Sets the enabling policy of transition \Code{t}$i$ to the
(normally marking-dependent) boolean \Code{b}$i$.
The default policy for a transition \Code{t} is \Code{disabled(t)},
that is, the firing time of a transition is restarted whenever it becomes
disabled by the firing of some other transition.
To specify that \Code{t} is ``restarted'' \cite{1993PNPM-DSPNs}
only when \Code{u} or \Code{v} fire, use the entry
\Code{t:(fires(u)|fires(v))}.
To specify a ``continue'' \cite{1993PNPM-DSPNs},
or ``age enabling'' \cite{Ajmone1989a}, behavior, for \Code{t}, use
the entry \Code{t:false}.
When transition \Code{t} itself fires, its firing time is always restarted,
so it is not necessary to explicitly add the condition \Code{fires(t)}
to the enabling policy of transition \Code{t}.
Of course, for memoryless distributions (exponential, geometric with unit
step in a \Code{pn} having an underlying discrete-time Markov chain,
or deterministically equal zero, i.e., immediate),
the enabling policy is irrelevant.
}

\item
\Code{void assert(proc bool b1, proc bool b2, ...);}\\
Defines a set of (usually) marking-dependent \Code{bool} expressions,
or ``assertions'', which must be true in each marking.
If, at any point in the execution, assertion \Code{b}$i$ does not hold,
{\smart} issues an error message and exits.

This is particularly useful to debug a model when using explicit state-space
exploration techniques.

\item
\Code{{place}~places();}\\
Returns the set of places of the current model 
(allowed inside a \Code{pn} block).

\item
\Code{{trans}~transitions();}\\
Returns the set of transitions of the current model
(allowed inside a \Code{pn} block).


\end{itemize}


\IGNORE{
Thus, the preselection priority relation must be acyclic:
\Code{prio(t:u,u:t)} is illegal.}
\IGNORE{
More subtle errors can arise as well.
For example, an arc \Code{p:t:if(enabled(u),0,1)} becomes illegal if
used in conjunction with a preselection priority \Code{t:u}, since to
determine whether \Code{t} is enabled we need to know the cardinality
of its input arc from \Code{p}, which is zero if \Code{u} is enabled, one
otherwise.
But, to determine whether \Code{u} is enabled, we need to know
whether \Code{t} itself is enabled, since \Code{t} has preselection priority
over \Code{u}.
}
\IGNORE{
I don't think this is done yet...
{\smart} takes particular care to check against any type of cyclic
dependency, hence a {\smart} language file is guaranteed to be semantically
well-defined if it compiles without errors.
}

\IGNORE{
The process underlying the \Code{ctmcSpn} \Code{cs1} remains a
continuous-time Markov chain even if,
instead of \Code{expo}, we use more complex predefined
phase-type distributions, such as \Code{erlang}, \Code{hypo}, \Code{hyper}
for the specification of one or more firing times.
We can even recall our continuous-time Markov chain \Code{c1}
defined in Section \ref{SEC:CTMC}
and set the distribution of the firing time of a transition in a
stochastic Petri net to the time to absorption of \Code{c1}.
For example, we could change the rate of transition \Code{v}
in the call of \Code{firing} for the stochastic Petri net \Code{cs1} to
\begin{lstlisting}
v:tta(c1(tk(p)*0.0004))
\end{lstlisting}

An exactly analogous discussion applies to discrete-time Markov chains
and stochastic Petri nets whose underlying process is a discrete-time
Markov chain: any DDP distribution can be used for the firing time of
the transitions \cite{1995StewartBook-DiscreteSPN}.
}

\end{private}
