% 
%

\chapter{The {\smart} language} \label{SEC:Language}
\markboth{THE SMART LANGUAGE}{}

This chapter introduces the {\smart} tool, its input language, and
its capabilities.
After an initial overview section, three main sections describe
how to define and manipulate deterministic quantities, random variables,
and stochastic processes, respectively.

%%%%%%%%%%%%%%%%%%%%%%%%%%%%%%%%%%%%%%%%%%%%%%%%%%%%%%%%%%%%%%%%%%%%%%
\section{Overview}

This section describes how to run {\smart} and the general structure of
{\smart} input files.
We adopt the convention of using a ``\Code{.sm}'' extension for the names
of these files, but this is not a requirement.

\subsection{Running {\smart}} 

Normally, {\smart} is run on one or more input files.
The command line used to run {\smart} on input file \Code{ff.sm} is
\lstset{language=bash}
\begin{lstlisting}
smart ff.sm
\end{lstlisting}
The input can also be partitioned over multiple files,
and it is possible to specify the standard input stream instead
of an ordinary file, by using ``\Code{-}''.
For example,
\begin{lstlisting}
smart ff1.sm ff2.sm - ff3.sm - ff4.sm
\end{lstlisting}
directs {\smart} to obtain input from file \Code{ff1.sm},
then file \Code{ff2.sm},
then the standard input (until the end-of-file is encountered),
then file \Code{ff3.sm}, then file \Code{ff4.sm}.
Note that the standard input stream is closed after finding the end-of-file,
thus the second \Code{-} specifier has no effect and files \Code{ff3.sm} and
\Code{ff4.sm} will not be able to read from standard input
(i.e., they cannot use functions that perform runtime input).


Running {\smart} with no arguments will display a help screen with version
information, as well as any command-line switches.
An important new feature of {\smart} is its built-in \emph{help} mechanism,
which may be invoked from the command line using the switch \Code{-h}.
For example,
\begin{lstlisting}
smart -h ctmc
\end{lstlisting}
will display documentation about the \Code{ctmc} formalism.

\subsection{Comments}

A {\smart} input file can contain comments.
Both single-line comments
\lstset{language=smart}
\begin{lstlisting}
int i := 1; // a single-line comment
\end{lstlisting}
and bracketed comments
\begin{lstlisting}
int i := 1; /* a bracketed
               comment */
\end{lstlisting}
can be used.
In the first case, the comment, starting from ``\Code{//}'' until the
end of the line, is ignored.
In the second case, the comment, starting from ``\Code{/*}'' until the
``\Code{*/}'', is ignored.


\subsection{\Code{\# include} directive}

{\smart} provides a file inclusion directive.
If a line
\begin{lstlisting}
#include "abc.sm"
\end{lstlisting}
appears anywhere in an input file (except within the quotes delimiting a string
or in a comment), the contents of file \Code{abc.sm} are read and used as if
they appeared in place of the \Code{include} line.

Inclusion directives can be recursively applied (i.e., an included file
can contain inclusion directives), as long as no circular
inclusions exist (e.g., if \Code{abc.sm} includes \Code{def.sm},
\Code{def.sm} cannot include \Code{abc.sm}).


\subsection{{\smart} statements}

A {\smart} input consists of a sequence of statements.
There are four classes of basic statements and two classes of
compound statements, which can be nested:
\begin{itemize}
\item
A {\bf \index{declaration}declaration} statement
is used to declare a \emph{function} of some set of \emph{parameters}.
As a special case, the set of parameters for a function can be empty,
thus the function is a constant.
In this sense, however, being ``constant'' should not to be confused with
being non-random (i.e., deterministic).
To ensure strict type-checking, the \emph{type} of the function and of
its parameters must be defined.  
\item
A {\bf \index{definition}definition} statement is used to
declare a function in the same
way declaration statements do, but it also specifies how to compute its value.
\item
An {\bf expression} statement is used to print the value of an expression,
although some function calls appearing in the expression
can also have side-effects,
such as generating output (see function \Code{print}),
or terminating the program (see function \Code{exit}).

\item
An {\bf option} statement is used to modify the behavior of {\smart}.
For example, there are options to control the solution algorithms (such
as the precision or the maximum number of iterations) or the level
of verbosity.
Options statements appear on a line beginning with ``\Code{\#}''
(the \Code{\# include} directive also begins with a ``\Code{\#}'', but it
should not be considered a statement).

\item
A \Code{for} compound statement is used to define arrays or to
repeatedly evaluate parametric expressions.
This is particularly useful for studies that explore how a result is
affected by a change in the modeling assumptions, such as the rate of
an event or the maximum size of a buffer.
\item
A \Code{converge} compound statement is used to specify a numerical fixpoint
iteration of the type often employed in conjuction with approximations.
\end{itemize}


\TBD{Rewrite this paragraph.}
The {\smart} language is strictly-typed, so objects belong to a \emph{type}:
\Code{void},
\Code{int},
\Code{bigint},
\Code{real},
\Code{bool}, or
\Code{string}.
All types are predefined, that is, dynamic definition of new types is not
possible, except for the creation of (multi-dimensional) arrays.
In addition to a type, an object also has a \emph{nature}, that is, it can
be a \emph{deterministic quantity} (the default),
a \emph{random variable}, or a \emph{stochastic process}.
For example, if we are modeling a computer system,
the number of processes waiting for the CPU as a function of time
is a stochastic process,
the number of processes waiting for the CPU at time 10 is a random
variable, and the average number of processes waiting for the CPU at time 10
is a deterministic quantity.
The following three sections describe how to specify deterministic quantities,
random variables, and stochastic processes.

%%%%%%%%%%%%%%%%%%%%%%%%%%%%%%%%%%%%%%%%%%%%%%%%%%%%%%%%%%%%%%%%%%%%%%
\section{Deterministic quantities}

The {\smart} language can be used to perform ``calculator-like'' computations.
It is important to become familiar with these simpler features of
the language before moving onto random variables and stochastic processes.

\subsection{A constant: $\pi$}

An example of a simple definition is
\begin{lstlisting}
real pi := 3.14;
\end{lstlisting}
which defines the constant \Code{pi} of type \Code{real} with value
\Code{3.14}.  Since the {\smart} language is declarative, not procedural,
the value of \Code{pi} is now set for the rest of its life.  A second
definition
\begin{lstlisting}
real pi := 3.1415;
\end{lstlisting}
causes an error.  Indeed, even a second occurrence of exactly the same
definition causes an error: each function can appear exactly once on
the left-hand side of a definition statement.
To query the value of \Code{pi}, we use an expression statement:
\begin{lstlisting}
pi;
\end{lstlisting}
which outputs the following line to the current output stream:
\lstset{language=output}
\begin{lstlisting}
3.14
\end{lstlisting}
\lstset{language=smart}
If we wanted a more informational output, we could format it
using the predefined function \Code{print}:
\begin{lstlisting}
print("The value of pi is ", pi, ".\n");
\end{lstlisting}
which would output the following line to the current output stream:
\lstset{language=output}
\begin{lstlisting}
The value of pi is 3.14.
\end{lstlisting}
\lstset{language=smart}






\subsection{A recursive one-parameter function: the factorial}
\label{SEC:factorial}

We can define a function to compute the factorial of an integer,
$n! = 1 \cdot 2 \cdot \ldots \cdot n$, as
\begin{lstlisting}
int fact(int n) := cond(n==0,1,n*fact(n-1));
\end{lstlisting}
The predefined function \Code{cond} returns the value of the second or
third parameter, according to whether the first parameter is \Code{true}
or \Code{false}, respectively.

We can call a function using either
\emph{named} parameter notation or with \emph{positional} parameter notation.
Thus, both \Code{fact(n:=7)} and \Code{fact(7)} evaluate to $7!$,
i.e., \Code{5040}.
The named parameter notation is especially useful to avoid errors and improve
readability when a function has many parameters.

\subsection{Arrays: the first ten factorial numbers}

The definition of function \Code{fact} does not
cause any computation to be performed.
To compute and output the value of the
first ten factorial numbers, we can use a \Code{for} statement:
\begin{lstlisting}
for (int i in {0..9}) {
  fact(n:=i);
}
\end{lstlisting}
but doing so requires the computation of each factorial number ``from scratch''.
A better way to do this is to use an array \Code{f} indexed by an
\emph{iterator} \Code{i}:
\begin{lstlisting}
for (int i in {0..9}) {
  int f[i] := cond(i==0,1,i*f[i-1]);
  f[i];
}
\end{lstlisting}
This defines an array \Code{f} with an index set from \Code{0} to \Code{9},
with a step of one (the default).  To refer to a particular value computed
inside a \Code{for} statement from a subsequent statement, the traditional
``square brackets'' notation is used.  
This definition is the most efficient, since it requires only nine
multiplications in total.
However, it computes only the first ten factorial numbers:
it does not help if we need to compute $12!$.

A fundamental observation is appropriate at this point.
Since {\smart} does not evaluate a function until its value is required in
output, our array-based definition of the first ten factorial numbers
is just as efficient when written as
\begin{lstlisting}
for (int i in {9..0..-1}) {
  int f[i] := cond(i==0,1,i*f[i-1]);
}
for (int i in {9..0..-1}) {
  f[i];
}
\end{lstlisting}
that is, if we were counting \Code{i} down from 9 to 0
(the step is explicitly specified to be \Code{-1} in
the above \Code{for} statements).
Note, however, that
\begin{lstlisting}
for (int i in {9..0..-1}) {
  int f[i] := cond(i==0,1,i*f[i-1]);
  f[i];
}
\end{lstlisting}
would result in an error message because it attempts to output
\Code{f[i]}, which requires the value of \Code{f[i-1]}, before
\Code{f[i-1]} has been defined.





\subsection{Managing large integers}

Ordinary \Code{int} quantities can only store quantities in a limited
range, $[-2^{31},2^{31}-1]$ on 32-bit architectures
and $[-2^{63}, 2^{63}-1]$ on 64-bit architectures.
In \smart, much larger integers have to be manipulated at times,
thus an ``arbitrary-size integer'' type, \Code{bigint}, is provided as well.
For example, if we define the two factorial functions
\begin{lstlisting}
bigint bfact(int n) := cond(n==0,1,bfact(n-1)*n);
int    ifact(int n) := cond(n==0,1,ifact(n-1)*n);
\end{lstlisting}
the \Code{for} statement
\begin{lstlisting}
for (int i in {1..20}) {
  print("ifact(",i:2,") = ",ifact(i):11,"    bfact(",i:2,") = ",bfact(i),"\n");
}
\end{lstlisting}
would produce, in output,
\begin{lstlisting}
ifact( 1) =           1    bfact( 1) = 1
ifact( 2) =           2    bfact( 2) = 2
ifact( 3) =           6    bfact( 3) = 6
...
ifact(10) =     3628800    bfact(10) = 3,628,800
ifact(11) =    39916800    bfact(11) = 39,916,800
ifact(12) =   479001600    bfact(12) = 479,001,600
ifact(13) =  1932053504    bfact(13) = 6,227,020,800
...
ifact(19) =   109641728    bfact(19) = 121,645,100,408,832,000
ifact(20) = -2102132736    bfact(20) = 2,432,902,008,176,640,000
\end{lstlisting}
%ifact( 1) =           1    bfact( 1) = 1
%ifact( 2) =           2    bfact( 2) = 2
%ifact( 3) =           6    bfact( 3) = 6
%ifact( 4) =          24    bfact( 4) = 24
%ifact( 5) =         120    bfact( 5) = 120
%ifact( 6) =         720    bfact( 6) = 720
%ifact( 7) =        5040    bfact( 7) = 5,040
%ifact( 8) =       40320    bfact( 8) = 40,320
%ifact( 9) =      362880    bfact( 9) = 362,880
%ifact(10) =     3628800    bfact(10) = 3,628,800
%ifact(11) =    39916800    bfact(11) = 39,916,800
%ifact(12) =   479001600    bfact(12) = 479,001,600
%ifact(13) =  1932053504    bfact(13) = 6,227,020,800
%ifact(14) =  1278945280    bfact(14) = 87,178,291,200
%ifact(15) =  2004310016    bfact(15) = 1,307,674,368,000
%ifact(16) =  2004189184    bfact(16) = 20,922,789,888,000
%ifact(17) =  -288522240    bfact(17) = 355,687,428,096,000
%ifact(18) =  -898433024    bfact(18) = 6,402,373,705,728,000
%ifact(19) =   109641728    bfact(19) = 121,645,100,408,832,000
%ifact(20) = -2102132736    bfact(20) = 2,432,902,008,176,640,000
on a 32-bit machine.
Note that the \Code{int} factorial overflows from \Code{13} on,
so that the returned values are incorrect, and can even be negative,
as \Code{ifact(20)} illustrates.
{\smart} does not raise an error when an overflow happens:
this is unfortunate, but it is common integer behavior
in modern programming languages.

Using a \Code{bigint} instead of an \Code{int} ensures that the
correct value is computed, but incurs a substantial overhead.
Thus, {\smart} uses ordinary integers for its normal internal operations,
and the \Code{bigint} type should be used only when needed.
It is the responsibility of the user to decide when this is the case.

\TBD{Need to discuss ``\Code{,}'' in formatting of \Code{bigint}}


\begin{comment}
\subsection{A \Code{real} iterator}

\TBD{Is this something we want to forbid from future versions?
In restrospect, it might be more trouble than it is worth it...}

Array iterators can be of type \Code{real}, as in the following
example defining a three-element array \Code{y}:
\begin{lstlisting}
for (real x in {O.1,O.2,0.4}) {
  real y[x] := 3*x*x;
}
\end{lstlisting}
Care must be taken to avoid indexing errors due to the finite precision of
floating-point arithmetic.  
To address this problem, {\smart} considers two floating-point indices to be
equal if they are within an absolute or relative precision $\delta$.
In the above example, {\smart} evaluates the expression \Code{y[x]} to 
\Code{y[0.1]} for any value of \Code{x} satisfying 
\[
  \begin{array}{c@{~~~~}l}
     0.1-\delta \leq \Code{x} \leq 0.1+\delta
     & 
     \mbox{if absolute precision is used,}
     \\
     0.1-0.1\delta \leq \Code{x} \leq 0.1+0.1\delta
     &
     \mbox{if relative precision is used.}
  \end{array}
\]
The value of $\delta$ is dictated by the option \Code{IndexPrecision}
(with a default of $10^{-5}$) and the
type of precision test is dictated by the option \Code{IndexPrecisionTest}
(with default value \Code{RELATIVE}, the other possible value
being \Code{ABSOLUTE}).
Thus, if the options are set using the statements
\begin{lstlisting}
# IndexPrecisionTest ABSOLUTE
# IndexPrecision  0.015
\end{lstlisting}
then the expression \Code{y[0.099]} evaluates to \Code{0.03}, and the
expression \Code{y[0.08]} results in a ``range error'' message,
and evaluates to \Code{null}.
\end{comment}





\subsection{Specifying numeric sets to be used in \Code{for} statements}

A numeric set of \Code{int} or \Code{real} values can be specified by explicitly
listing its elements, e.g., \verb%{3,4,7}% or \verb%{0.3,0.4,0.7}%.
The type of a set defined this way is \verb%{int}% if each of its elements
is an \Code{int}, \verb%{real}% if each of its elements is a \Code{real}.
No mixing of \Code{int} and \Code{real} elements is allowed.  

\begin{comment}
The type of a set defined this way is \verb%{int}% if each of its elements
is an \Code{int}, \verb%{real}% if any of its elements is a \Code{real}.
Thus, for example, \verb%{3,4,7.0}% is of type \verb%{real}%.
\end{comment}

It is also possible to specify a range of elements, instead of a single
element,  when defining a set.
A range has the syntax \Code{$first$..$upper$..$step$} and is equivalent
to explicitly specifying the elements $first$, $first+step$,
$first+step+step$, and so on, up to, but not included, the first element
that exceeds $upper$.
Exceptions: if $step$ evaluates to $0$, the range identifies the single
element $first$ while, if $first > upper$ and $step > 0$,
or $first < upper$ and $step < 0$, the range identifies no elements at all;
in either case {\smart} issues a warning message.
The type of the elements identified by a range
is \Code{real} if $first$, $upper$, or $step$ is \Code{real},
it is \Code{int} otherwise.
When $first$ and $upper$ have type \Code{int}, $step$ can be omitted:
it defaults to \Code{1}, thus \Code{3..7} is equivalent to \Code{3..7..1},
they both identify the elements \Code{3,4,5,6,7}.
However, it is an error to omit the $step$ when $first$ or $upper$
have type \Code{real}.
It is possible to mix the two modes of set specification, as in
\verb%{1,3,5..10,20..50..5}%.
Care must be taken when using \Code{real} ranges, due to the finite
precision of floating point numbers:
\Code{\{0.1..0.9..0.1\}} might specify either eight or nine \Code{real}
elements, depending on whether the result of summing
nine values \Code{0.1} is greater than \Code{0.9} or not,
when these quantities are represented as floating-point numbers
(one could specify \Code{\{0.1..0.91..0.1\}} to ensure that the range
contains exactly nine elements).
\begin{comment}
As expected, the type of the set is \verb%{real}% if any of the elements
specified has type \Code{real}, otherwise it is \verb%{int}%.
\end{comment}

While not, strictly speaking, a set property, the order in which the elements
are specified in the set is important in {\smart};
for example, it affects the order in
which a \Code{for} statement is executed.
In turn, this order can affect correctness or efficiency.
{\smart} respects the order specified by the user, so, for example,
\verb%{-3..0}% and \verb%{0..-3..-1}% represent the same set,
but with the elements in reverse order.
Note that it is not an error to specify the same element multiple times
in a set, but any subsequent specification is ignored.
Thus, the two set specifications \verb%{1,3,2..6,4,8}% and
\verb%{1,3,2,4,5,6,8}% are exactly equivalent, with respect to both the
elements they contain and their order.

\TBD{We might want to drop sets of reals...}

\TBD{Discuss: braces are also used to collect void expressions into sequences.}




\subsection{Nested \Code{for} statements}

Since \Code{for} statements can be nested, the dimensionality of a
declaration is determined by the sequence of iterators, from the
outermost to the innermost.  For example,
\begin{lstlisting}
for (int n in {1,2}) {
  for (int k in {1..5}, int i in {3,7,5}) {
    real g[n][k][i] := 0.2*n*k*i;
  }
}
\end{lstlisting}
defines a tridimensional ($2 \times 5 \times 3$)
array \Code{g} of \Code{real} constants.
Note that the scope of the iterators in a \Code{for} statement is local to that
statement and to those included in it.

It is an error to define objects with a smaller dimensionality than that
corresponding to the enclosing \Code{for} statements, or with indices
in an order other than that in which they appear as iterators
Thus, the definition of \Code{h} in
\begin{lstlisting}
for (int i in {0..9}, int j in {0..9}) {
  int h[i] := ...;
  int f[i][j] := ...;
  int g[j][i] := ...;
}
\end{lstlisting}
is illegal because it does not have index \Code{j}, while the definition
of \Code{g} is illegal because index \Code{i} follows index \Code{j}.

Analogously, it is illegal
to define functions with parameters within a \Code{for} statement,
such as \Code{fact} in the following:
\begin{lstlisting}
for (int i in {0..9}) {
  int fact(int n) := cond(n==0,1,n*fact(n-1));
  int f[i] := fact(i);
}
\end{lstlisting}
This is because such a function would be either independent of the
iterators (as \Code{fact} is in the above example), or
dependent on them;
in the former case, it could be specified outside the \Code{for} statement,
in the latter the semantic of calling it from outside the \Code{for}
statement would be undefined, since the iterators have no value
outside the \Code{for} statement.


\subsection{Parameter defaults}

Normally, each formal parameter in the definition of a function must
correspond to an actual parameter in a function call.
However, it is possible to set default values for the formal parameters.
If we define
\begin{lstlisting}
int f2(int a:=1, int b) := a*b;
\end{lstlisting}
the function calls \Code{f2(a:=1,b:=3)}, \Code{f2(1,3)},
\Code{f2(a:=default,b:=3)}, \Code{f2(default,3)}, or \Code{f2(b:=3)} are
exactly equivalent.
However, the call \Code{f2(3)} is illegal: named
parameters must be used if some parameters are not listed explicitly.

Note that it is possible to define a default value for each parameter
of a function, but it is not possible to call that function with an
implicit default for all of its parameters
(i.e., at least one of them must be set to \Code{default} explicitly),
since the function must be invoked with named parameters.
\TBD{The above statement is now false.  It is a bit ugly, perhaps, but
the function call \Code{f()} will use all default values,
while \Code{f} will not.
Perhaps worse, a ``constant'' can be called using either \Code{pi} or \Code{pi()}.
In any event, the rule is that
\Code{f()} uses named parameter passing, while \Code{f} uses positional 
parameter passing.}

\begin{comment}
\subsection{A parallelizable \Code{for} statement}

The \Code{for} statement is a natural candidate for parallelization.
Indeed, {\smart} can recognize parallelism and perform
the concurrent computation of all the values specified in a \Code{for}
statement, provided no dependencies from one ``iteration'' to the next exist,

For example, the 100 values
\begin{lstlisting}
for (int i in {0..9}, int j in {0..9}) {
  int measure[i][j] := mymodel(i,j);
}
\end{lstlisting}
can be computed concurrently,
This reduces the total execution time in cases where the solution of
\Code{mymodel}
requires substantial computation, and multiple processors are available.
Such situations are common in parametric modeling studies.
\end{comment}

\subsection{Overloading: a \Code{real} factorial} \label{SEC:realFactorial}

The {\smart} language allows identifier overloading.
The same identifier can be used to define multiple functions, as long as
the order, type, or name of the formal parameters can be used to
distinguish which one is meant by a function call.  For example, we
could define both an ``integer factorial'' and a ``real factorial'':
\begin{lstlisting}
int  fact(int  n) := cond(n<1,1  ,n*fact(n-1));
real fact(real n) := cond(n<1,1.0,n*fact(n-1));
\end{lstlisting}
Then, \Code{fact(8)} and \Code{fact(5.4)} would call the standard
(integer) factorial or the real factorial, respectively.

\begin{comment}
Overloading also applies to arrays.  If we defined
\begin{lstlisting}
for (int i in {0..9}) {
  real f[i] := i * 1.0;
}
for (real i in {1.0..9.0..1.0}) {
  real f[i] := i * 10.0;
}
\end{lstlisting}
the expressions \Code{f[2]} and \Code{f[2.0]} would refer to the first and
second array, respectively.
\end{comment}
Furthermore, we can define arrays and functions with the same name;
these are distinguished at the syntactical level by the use of
``\Code{[]}'' or ``\Code{()}'', respectively.

As always, common sense should be followed:
overloading can be helpful when used judiciously, but it can become confusing
when overused, and it can lead to ambiguous function calls altogether.
For example, if we have defined the functions
\begin{lstlisting}
int f(real y) := ... ;
int f(real x) := ... ;
\end{lstlisting}
we can distinguish between them with the named parameter calls
\Code{f(x:=1.0)} and \Code{f(y:=1.0)}, but {\smart} rejects the positional
parameter call \Code{f(1.0)} as illegal.
Conversely, the functions
\begin{lstlisting}
int f(real x, int i) := ... ;
int f(int i, real x) := ... ;
\end{lstlisting}
cannot be distinguished if we use the named parameter call
\Code{f(x:=3.1,i:=1)}, but can be
distinguished if we use the positional calls \Code{f(3.1,1)}
and \Code{f(1,3.1)}.
For analogous reasons, the functions
\begin{lstlisting}
int f(int i:=1) := ... ;
int f(real x:=2) := ... ;
\end{lstlisting}
cannot be called using \Code{f(default)}.
Finally, the type returned by a function cannot be used to resolve ambiguities
due to overloading, thus, when {\smart} reads the second definition for \Code{f}
in
\begin{lstlisting}
int f(int x):= ... ;
real f(int x):= ... ;
\end{lstlisting}
it will issue an error message and ignore it.




\subsection{Type promotions: manual or automatic?}

\TBD{Update this; especially to include automatic promotions
of functions (adding proc and rand as necessary)}

\begin{table}
\renewcommand{\arraystretch}{0.97}
\begin{center}\begin{tabular}{| l | l |}
\hline
\multicolumn{1}{| c |}{\bf Original type} &
\multicolumn{1}{  c |}{\bf Promoted type} \\
\hline
\hline
\Code{int              }&\Code{real}\\
\Code{int              }&\Code{bigint}\\
\Code{bigint           }&\Code{real}\\
\Code{rand int         }&\Code{rand real}\\
\Code{proc int         }&\Code{proc real}\\
\hline
\Code{bool             }&\Code{rand bool}\\
\Code{int              }&\Code{ph int}\\
\Code{ph int           }&\Code{rand int}\\
\Code{real             }&\Code{rand real}\\
\Code{ph real          }&\Code{rand real}\\
\hline
\Code{rand bool        }&\Code{proc bool}\\
\Code{rand int         }&\Code{proc int}\\
\Code{rand real        }&\Code{proc real}\\
\hline
\Code{proc bool        }&\Code{bool}\\
\Code{proc int         }&\Code{int}\\
\Code{proc real        }&\Code{real}\\
\hline
\end{tabular}\end{center}
\caption{Automatic type promotions. \TBD{Needs to be expanded.}}
\label{TAB:typePromotions}
\end{table}

Assume that we defined an integer constant \Code{k}, and we now want
to call the real factorial function with \Code{k} as the actual
parameter.  One reason for wanting to do this is that, if \Code{k} is
large, the integer factorial function might result in an overflow,
while the value real factorial might still fit in the floating point
representation of the machine.

We can force an automatic conversion by summing a (real) zero to
\Code{k}, as in \Code{fact(k+0.0)}.  This method works, but is inelegant;
a much better way is to use an explicit type conversion, \Code{fact(real(k))}.

For simplicity's sake, promotions such as the \Code{int}-to-\Code{real}
promotion for \Code{int k} in the expression \Code{k+0.0}, occur automatically.
Table \ref{TAB:typePromotions} lists the automatic
type promotions that can take place when evaluating an expression.
Other promotions must be explicitly requested
using the syntax \Code{new\_type(variable)}, where it must be possible
to promote the type of \Code{variable} to \Code{new\_type}.
In addition to the automatic promotions of Table \ref{TAB:typePromotions},
the legal explicit promotions are
from \Code{bigint} to \Code{int} and from \Code{real} to \Code{int}
or to \Code{bigint}.
Note that \Code{int(x)} and \Code{bigint(x)} discard the decimal portion of
the real value \Code{x} while, if rounding to the closest integer is desired,
\Code{round(x)} should be used instead.
Functions \Code{floor} and \Code{ceiling} are also available.
For example,
\Code{int(3.2)},
\Code{floor(3.2)}, and
\Code{round(3.2)} evaluate to \Code{3}, while
\Code{ceiling(3.2)} evaluates to \Code{4};
\Code{int(-3.2)},
\Code{round(-3.2)},
and \Code{ceiling(-3.2)}
evaluate to \Code{-3}, while
\Code{floor(-3.2)}, evaluates to \Code{-4}.





We observe that the \Code{real}-to-\Code{int} and \Code{bigint}-to-\Code{int}
conversions cause an overflow if the value to be converted does not fit
in an \Code{int} type;
indeed, even the \Code{bigint}-to-\Code{real} conversion can overflow,
since a \Code{bigint} can have thousands of digits 
({\smart} issues a warning message in this case, and the value of
the converted expression is set to \Code{null}).

Also, casting an \Code{int} or \Code{bigint} into a \Code{real}
may cause loss of precision, since the number of bits for the mantissa
is fixed to a relatively small number in the internal representation
of a \Code{real} (of course, no warning is issued in this case).

Promotion of the actual parameters in a function call or of the
expressions used to index an array, however, requires special
attention, because of the possible ambiguities arising from
overloading.  Consider for example the definitions
\begin{lstlisting}
real x(int i, real j) := ... ;
real x(real i, int j) := ... ;
\end{lstlisting}
The function calls \Code{x(1,1.0)} and \Code{x(1.0,1)} clearly refer to
the first and second definitions, respectively.  The call \Code{x(1,1)},
however, could refer to either, depending on whether the first or the
second actual parameter is promoted to \Code{real}.
{\smart} first attempts to find a function or array requiring no promotion at
all, that is, a function or array for which the formal parameters or indices
exactly match the actual parameters or indices used in the expression.
Only if this fails is promotion attempted.
If there are multiple choices for promoting the set of parameters,
{\smart} searches for a unique choice that requires promoting the minimum
number of parameters: if there is such a function or array, it is chosen;
otherwise, {\smart} issues an error message.
The option \Code{ParameterWarnings} controls under what conditions a warning is
issued when {\smart} automatically promotes parameters in function calls
(see section \ref{SEC:smart-options}).

\begin{comment}
Finally, note that a set of integers, \Code{\{int\}}, cannot be promoted
(neither automatically nor explicitly) to a set of reals, \Code{\{real\}}.
Thus, in the following, the former works but the latter is an error:
\begin{lstlisting}
for (real r in {1..5..1.0}) { ... }
for (real r in {1..5..1})   { ... }
\end{lstlisting}
\end{comment}


\subsection{Using iterator values to define iterators}

The value of an iterator can be used to define the value of further inner
iterators:
\begin{lstlisting}
for (int x in {1..9}, int y in {x*2,x**,x*8}) {
  int v[x][y] := x*y;
}
\end{lstlisting}

Syntactically, the iterators of a \Code{for} statement must always be
scalars: \Code{y} depends on \Code{x} in the above examples, but its
dependency is implicit.
In other words, it is an error to write
\begin{lstlisting}
for (int x in {1..9}, int y[x] in {x*2,x*4,x*8}) {
  int v[x][y[x]] := x*y[x];
}
\end{lstlisting}

\subsection{The \Code{null} value}

Our real factorial function defined in Section \ref{SEC:realFactorial} 
returns \Code{1} if it is called with a negative argument.
The integer factorial defined in Section
\ref{SEC:factorial} has an even worse behavior: it goes into an
infinite recursion if it is called with a negative argument.  
To prevent these problems, we can use the special value \Code{null}, and
modify our factorials as follows:
\begin{lstlisting}
int  fact(int n)  := cond(n<0,null,cond(n==0,1,  n*fact(n-1)));
real fact(real n) := cond(n<0,null,cond(n<1, 1.0,n*fact(n-1))); 
\end{lstlisting}

Any expression where one of the operands is \Code{null},
such as \Code{null+3} or \Code{sqrt(fact(-2))} also evaluates
to \Code{null}.
The only way an expression evaluating to \Code{null} can be meaningfully
used is if it is used as the argument of a function call, and the function
knows how to deal with a \Code{null} value.
The predefined function \Code{cond}, for example,
simply returns the value of the second or third parameter; if the
selected one is \Code{null}, so is the returned value.  The same holds
for the predefined function \Code{case}.
The predefined function \Code{is\_null} is also available:
it returns \Code{true} or \Code{false}
according to whether its argument has value \Code{null} or not.

Some expressions or functions might also evaluate to \Code{null}.
For example, \Code{1/0} is undefined because we cannot determine its
sign (see the next section, which discusses infinite values), thus
it evaluates to \Code{null}.
Analogoulsy, \Code{pow(-3,2.5)} is undefined because the base of the power
function must be non-negative if the exponent is real, thus it also
evaluates to \Code{null};
however, \Code{pow(-3,2)} is legal and evaluates to \Code{9}.

\subsection{Managing \Code{infinity}}

{\smart} allows an \Code{int} or \Code{real} function to have an infinite value.
The predefined \Code{int} value \Code{infinity} can be used for this purpose.
The sign of \Code{infinity} is positive, thus we can express a negative
infinite value by negating it.
Since \Code{int} values can be promoted to \Code{real}, real
values can have an \Code{infinity} value as well:
\begin{lstlisting}
int positive_infinity := infinity;
int negative_infinity := -infinity;
real x := negative_infinity;
\end{lstlisting}
Operations involving \Code{infinity} are defined as expected.
For example, the following expressions are legal:
\begin{lstlisting}
-infinity + infinity * (-5) + 72;      // -infinity (of type int)
-infinity / -3.4;                      // infinity (of type int)
infinity > -infinity;                  // true
pow(3,7.2*infinity);                   // infinity (of type real)
\end{lstlisting}
However, operations such as the following ones are undefined
({\smart} returns \Code{null} and issues a warning):
\begin{lstlisting}
infinity - infinity;
0 * infinity;                     
infinity >= infinity;
3.2/0;
pow(-3,infinity);
mod(3,infinity);
mod(infinity,3);
\end{lstlisting}
When printing an \Code{int} or \Code{real} value that evaluates to
\Code{infinity}, {\smart} uses the value of the option \Code{InfinityString},
which has a default value of \Code{"infinity"}.
This affects only the way an infinite value is written to output,
not the way it must occur in input.
Thus, for example, the input file
\begin{lstlisting}
# InfinityString "oo"
print(0,",",1,",",2,",...,",infinity,"\n");
\end{lstlisting}
produces the output
\begin{lstlisting}
0,1,2,...,oo
\end{lstlisting}


\subsection{Fixpoint iterations: solving $x = \sqrt{x}$}

An important feature of {\smart} is its ability to use numerical fixpoint
iterative solutions, where multiple parametric models are solved repeatedly.
The approach starts with initial guesses for their parameters, and successively
refines them, until convergence within a given precision is achieved,
or a maximum number of iterations is reached.
For example, we can find the zero $x=1$ for the expression
$x - \sqrt{x}$ by transforming it into the fixpoint equation $x = \sqrt{x}$,
and starting with any positive guess for $x$:
\begin{lstlisting}
converge {
  real x guess 100;
  real x := sqrt(x);
}
\end{lstlisting}
The value of \Code{x} after the \Code{converge} statement is the computed
fixpoint value.  The precision $\epsilon$ to which this value is
computed can be changed in an option statement.  The iterations are
stopped when two subsequent values for \Code{x} are within a relative or
absolute precision of $\epsilon$ (note that this does not guarantee that 
\Code{x} is within $\epsilon$ of the true fixpoint solution), or when the
number of iterations exceeds the maximum number of allowed iterations 
defined by the option \Code{MaxConvergeIters}.
Similar to the case of array index precision, the converge precision $\epsilon$ 
and the type of precision used are determined by the options 
\Code{ConvergePrecision} and \Code{ConvergePrecisionTest}.
Note that the statement providing an initial guess for \Code{x}
is a declaration: the definition of \Code{x} must also appear in the
body of the \Code{converge} statement.
However, unlike ordinary declarations, at most one \Code{guess} statement
can appear for each \Code{x} before the definition of \Code{x},
and none after.
Only \Code{real} objects can be defined inside a \Code{converge}
statement.

If any variable being iterated upon by the \Code{converge} statement becomes
\Code{null}, it will remain \Code{null} regardless of the values it might
be assigned in further iterations.
Thus, for example, if we wanted to find the zero of $x - \sqrt{x+1}$
but misstyped ``\Code{x-1}'' instead of ``\Code{x+1}'' in the following
\begin{lstlisting}
converge {
  real x guess 10;
  real x := sqrt(x-1);
}
\end{lstlisting}
the subsequent values assumed by \Code{x} would be
\Code{3}, which is  \Code{sqrt(10-1)}, then
\Code{1.41421}, which is \Code{sqrt(3-1)}, then
\Code{0.64359}, which is \Code{sqrt(1.41421-1)}, and finally
\Code{null}, since \Code{sqrt(0.64359-1)} is undefined.

We stress that, unlike the \Code{for} statement, the \Code{converge}
statement does not affect the dimensionality of the objects declared in it.



\subsection{Multi-variable fixpoint iterations} \label{SEC:multconv}

In practice, most fixpoint iterations involve several variables and
complex dependencies among them.
Consider the fixpoint iteration
\begin{lstlisting}
converge {
  real d guess 0.5;
  real b guess 0.3;
  real a guess 0.2;
  real c guess 0.7;
  real c := fc(d,b);
  real b := fb(d,c);
  real a := fa(b,c);
  real d := fd(a);
}
\end{lstlisting}
The dependency graph of Fig.~\ref{FIG:NestedDep} represents this
situation.  An arc from $x$ to $y$ signifies that the value of $y$
depends on the value of $x$.  

In this \Code{converge} statement, iterations continue until
subsequent values for \Code{a}, \Code{b}, \Code{c}, and \Code{d} differ by
less then $\epsilon$.
If the option \Code{UseCurrent} is set to \Code{true} (this is the default),
the newly computed values are used immediately, otherwise,
the newly computed values are not used until the next iteration.  
In our example, the computation of \Code{b}, \Code{a}, and \Code{d} is
affected by this choice:
with \Code{UseCurrent} set to \Code{true}, the calls
\Code{fb}, \Code{fa}, and \Code{fd} use the most recent values
available for their arguments.

{\smart} determines whether enough guesses are provided to start the iterations.
The rule is simple: any variable appearing on the right
side of an assignment statement must have a value when the statement
is executed.  If such a variable has not yet been defined, then it
must be given an initial guess.  If the option \Code{UseCurrent} is set
to \Code{false}, then every variable appearing on the right side of an
assignment must be given an initial guess.
Thus, in our example, the initial guesses for \Code{a} and \Code{c} are
required when \Code{UseCurrent} is set to \Code{false}, while 
they are not necessary (and are ignored) when \Code{UseCurrent}
is set to \Code{true}.

\subsection{Nested fixpoint iterations}

It is often advisable to nest fixpoint iterations.
For instance, in the \Code{converge} statement shown in the previous
section, we might know that \Code{b} and \Code{c}
require many iterations to stabilize, but that, fortunately, each call
to \Code{fb} and \Code{fc} requires little computation.  Then, the
following nested scheme might be advantageous:
\begin{lstlisting}
# UseCurrent true
converge {
  real d guess 0.5;
  real b guess 0.3;
  converge {
    real c := fc(d,b);
    real b := fb(d,c);
  }
  real a := fa(b,c);
  real d := fd(a);
}
\end{lstlisting}
Now, the values of \Code{b} and \Code{c} are iterated upon
until they stabilize, for each updated guess for \Code{d}.

Note that the position of the statement \Code{real b guess 0.3} causes
the guess \Code{0.3} to be used only once: the first time the innermost
\Code{converge} statement is executed.  The subsequent inner fixpoint
iterations performed whenever the value of \Code{d} is updated will use the
latest value for \Code{b}, not \Code{0.3}.  The alternative behavior, where
\Code{b} is set to \Code{0.3} every time the innermost \Code{converge}
statement is reached, can be obtained by moving the statement
\Code{real b guess 0.3} inside the innermost \Code{converge} statement,
but this would probably worsen the convergence behavior.

\begin{figure}
  \centering
  \includegraphics[scale=0.75]{figures/NestedDep.pdf}
  \caption{A \Code{converge} dependency graph.}
  \label{FIG:NestedDep}
\end{figure}

\subsection{Nested \Code{for} and \Code{converge} statements}

A \Code{for} statement can contain a \Code{converge} statement, as in
\begin{lstlisting}
for (int i in {0..5}) {
  converge {
    real x[i] guess 100;
    real x[i] := sqrt(x[i] + i);
  }
}
\end{lstlisting}
This specifies the computation of the solutions to the six fixpoint
equations $x = \sqrt{x+i}$, for $i \in \{0, \ldots, 5\}$.
Consider now the statement
\begin{lstlisting}
for (int i in {0..5}) {
  converge {
    real x[i] guess 100;
    real x[i] := cond(i==0,sqrt(x[0]),sqrt(x[i-1]+i));
  }
}
\end{lstlisting}
This computes the fixpoint solution \Code{x[0]=1}, as expected.
However, for any other value of \Code{i}, no fixpoint iteration is
specified.  For example, when \Code{i} equals one, the ``fixpoint
iteration'' is
\begin{lstlisting}
real x[1] := sqrt(x[0] + 1);
\end{lstlisting}
This simply sets \Code{x[1]} to $\sqrt{2}$, since the value of \Code{x[0]}
is now set to 1 (possibly with a small error, since this value is
computed numerically with an iterative procedure).
The use of previously computed values in a \Code{converge} statement
can be meaningful, especially when specifying an initial guess.
For example,
\begin{lstlisting}
for (int i in {0..5}) {
  converge {
    real x[i] guess cond(i==0,100,x[i-1]);
    real x[i] := sqrt(x[i] + i);
  }
}
\end{lstlisting}
uses the converged value of \Code{x[i-1]} as initial guess for \Code{x[i]}.
This is reasonable, since it could be expected that the value of the
fixpoint changes incrementally as \Code{i} goes from \Code{0} to \Code{5}.
Only for the first one, \Code{x[0]}, do we really need to provide a
``wild'' guess.

A \Code{converge} statement can contain a \Code{for} statement.  This is
useful when an array has cyclic dependencies.
For example, the linear system 
$$
\left\{ \begin{array}{r @{~=~} l}
5y_1 + y_{10} &  15 \\
y_{i-1} + 5y_i & 6i-1  ~~~ \mbox{for}~  i \in \{2, \ldots, 10\} ,
\end{array} \right.
$$
whose solution is $y_i = i$ for $i \in \{1, \ldots, 10\}$,
can be solved using the statement
\begin{lstlisting}
converge {
  for (int i in {1..10}) {
    real y[i] guess 0;
    real y[i] := cond(i==1, (15-y[10])/5, (6*i-1-y[i-1])/5);
  }
}
\end{lstlisting}

\subsection{Declarations and recursion: the convolution algorithm}

{\smart} allows for both declarations and definitions of functions.
Any number of declarations
\begin{lstlisting}
int func(int n, real x);
\end{lstlisting}
can appear anywhere in the input file, but the definition of this function,
\begin{lstlisting}
int func(int n, real x) := ... ;
\end{lstlisting}
must eventually appear exactly once.
The number, type, name, and order of the parameters in all the
declarations and in the definition must match (otherwise they are
assumed to refer to different, overloaded functions).

Declarations accomplish the same goal as the ``forward'' keyword in
Pascal or the templates of ANSI C.
While the use of declarations might be \emph{desirable} for clarity
(e.g., we might place them at the top of the input file,
postponing the definition of their values to a later section), it is
\emph{essential} when two functions refer to each other recursively.

As an example, we now consider the computation of the
normalization constant in a single-class closed product-form queueing
network with load-dependent servers
\cite{Jain1991} (readers unfamiliar with queueing networks may safely
skip the rest of this section).
Using the notation of
\cite{Trivedi2002book}, assume that there are $c_i$ servers with service
rate $\mu_i$ at node $i \in \{0, \ldots, m\}$, with relative visit
ratio $v_i$.  Then define
$$
   \rho_i = v_i / \mu_i ,  ~~~~~~~~~~~~~
\beta_i(k) = \left\{ \begin{array}{l l}
   k!             & \mbox{if $k < c_i$} \\
   c_i!c_i^{k-c_i} & \mbox{if $k \geq c_i$}
               \end{array}\right. , %\}
  ~~~~~~~~~~~~~
r_i(k) = \left\{ \begin{array}{l l}
   \rho_i^k / \beta_i(k)             & \mbox{if $k > 0$} \\
   1                                 & \mbox{if $k = 0$}
               \end{array}\right. . %\}
$$
If the total customer population is $n$, the joint probability of
having $k_i$ customers at node $i \in \{0, \ldots, m\}$ is
$$p(k_0, \ldots, k_m) = \frac{1}{C(n)} \prod_{i=0}^{m}
  \frac{\rho_i^{k_i}}{\beta_i(k_i)} ,$$
where the computation of the normalization constant
$$C(n) = \sum_{k_0+\cdots+k_m = n}\left( \prod_{i=0}^{m}
  \frac{\rho_i^{k_i}}{\beta_i(k_i)} \right) $$
can be performed recursively defining
$$
C_i(j) = \left\{ \begin{array}{l l}
   r_0(j)             & \mbox{if $i = 0$} \\
   \displaystyle \sum_{k=0}^j C_{i-1}(j-k) r_i(k) & \mbox{if $i \neq 0$}
               \end{array}\right. , %\}
$$
so that $C(n) = C_m(n)$.

Let us now show how to define the value of $C(n)$ in the {\smart} language.
The inputs $m$, $n$, $c$, $v$, and $\mu$ are read first,
using the two functions \Code{read\_int} and \Code{read\_real}, which read an
\Code{int} or a \Code{real} from the current input stream, respectively:
\begin{lstlisting}
int m := read_int("m");
int n := read_int("n");
for (int i in {0..m}) {
  int  c[i]  := read_int("c[i]");
  real v[i]  := read_real("v[i]");
  real mu[i] := read_real("mu[i]");
}
\end{lstlisting}
Then, we can define $\rho$, $\beta$, and $r$:
\begin{lstlisting}
real rho(int i) := v[i]/mu[i];
real beta(int i, int k) := cond(k<c[i],fact(real(k)),fact((c[i]))*pow(c[i],k-c[i]));
real r(int i, int k) := cond(k==0,1,pow(rho(i),k)/beta(i,k));
\end{lstlisting}
Finally, we can define $C$ with the following statements:
\begin{lstlisting}
real C(int i, int j);
real sum(int i, int j, int k) := cond(k==0,C(i,j),sum(i,j,k-1)+C(i,j-k)*r(i,k));
real C(int i, int j) := cond(i==0,r(0,j),sum(i-1,j,j));
\end{lstlisting}
Note that no computation is performed, since $C$ is defined as a
function, not an array.  Normally, we want to save the last column of
the tableau $C_i(j)$, hence we could define an array as follows:
\begin{lstlisting}
for (int j in {0..n}) {
  real C[j] := C(m,j);
}
\end{lstlisting}

The above definition of $C$ as a function is correct but extremely
inefficient, because it causes {\smart} to recompute the value $C(i,j)$
every time it is needed.  The complexity is reduced to $O(n^2m)$ if
the values $C(i,j)$ are stored in a table that allows direct lookup.
Hence, we should change the definitions of \Code{C} (for both the
function and the one-dimensional array) and \Code{sum} as follows:
\begin{lstlisting}
real sum(int i, int j, int k);
for (int i in {0..m}, int j in {0..n}) {
  real C[i][j] := cond(i==0,r(0,j),sum(i-1,j,j));
}
real sum(int i, int j, int k) := cond(k==0,C[i][j],sum(i,j,k-1)+C[i][j-k]*r(i,k));
\end{lstlisting}
Now, the entries \Code{C[i][j]} for \Code{i} equal \Code{m} contain the
desired final column of the tableau, and there is no reason to save it
as done before.  To increase the efficiency, $r$, $\rho$, and $\beta$
should be arrays as well.  Also, we have ignored overflows and
underflows in this example, although we did set the type of $\beta$ to
\Code{real} instead of \Code{int}, and used the real factorial, to
postpone the occurrence of an overflow.


\subsection{Strings}

For convenience, the classic \Code{string} type is defined in {\smart} as well.
One can define functions returing strings of characters, and use them
just like any other function.
A string constant in a {\smart} input file must be enclosed in quotes:
\begin{lstlisting}
string toolname := "SMART";
\end{lstlisting}
but, of course, the quotes are not printed in output, thus
\begin{lstlisting}
toolname;
\end{lstlisting}
produces simply
\begin{lstlisting}
SMART
\end{lstlisting}
Most strings are used to properly format output.
For example,
\begin{lstlisting}
for (int i in {1..30}) { print(i:2,cond(mod(i,10)==0,"\n","  ")); }
\end{lstlisting}
produces
\begin{lstlisting} 
 1   2   3   4   5   6   7   8   9  10
11  12  13  14  15  16  17  18  19  20
21  22  23  24  25  26  27  28  29  30
\end{lstlisting}

Special characters can be entered in a string by escaping them:
a newline character is entered as \verb%\n%;
a tab     character is entered as \verb%\t%;
a quote   character is entered as \verb%\q%;
a slash   character is entered as \verb%\\%; and
a bell (alarm)    character is entered as \verb%\a%.



%%%%%%%%%%%%%%%%%%%%%%%%%%%%%%%%%%%%%%%%%%%%%%%%%%%%%%%%%%%%%%%%%%%%%%
\section{Random variables} \label{SEC:RandomVars}

{\smart} can understand and manage random variables with discrete-time or
continuous-time phase-type distributions (DDP or DCP).
In \smart, these are defined using the types \Code{ph int} or \Code{ph real},
respectively.
Special operations are possible on them, and both numerical and Montecarlo
simulation algorithms can be applied.
Random variables with more general distributions are
identified as \Code{rand int} or \Code{rand real}, but they can
be manipulated only in restricted ways, although simulation is always
applicable.
For the essential definitions related to phase-type distributions,
see Appendix~\ref{SEC:PhaseType}.
\begin{release}
Currently, {\smart} does not support Montecarlo simulation.
\end{release}

\subsection{Discrete phase-type random variables}

In the {\smart} language, random variables with a DDP distribution
have type \Code{ph int}.
Such variables can be defined using predefined functions, or can be built
from other DDP random variables using various operators.
Also, any positive finite \Code{int} value can be promoted to a \Code{ph int}.
Given
\begin{lstlisting}
ph int X := geometric(0.7);
ph int Y := equilikely(1,5);
\end{lstlisting}
we can define:
\begin{lstlisting}
ph int sumXY    := X+Y;
ph int prodNX   := 4*X;
ph int chooseXY := choose(X:3,Y:7);
ph int minXY    := min(X,Y);
ph int maxXY    := max(X,Y);
ph int geomX    := geom(0.1,X);     // Not implemented currently
\end{lstlisting}
However, definitions such as:
\begin{lstlisting}
ph int diffXY   := X-Y;
ph int sumRX    := 4.0+X;
ph int prodRX   := 4.0*X;
ph int prodXY   := X*Y;
\end{lstlisting}
are illegal, since DDP distributions are not closed under these
operations.  A correct definition would be
\begin{lstlisting}
rand int  diffXY   := X-Y;
rand real sumRX    := 4.0+X;
rand real prodRX   := 4.0*X;
rand int  prodXY   := X*Y;
\end{lstlisting}

A note about stochastic independence: every time a random variable is defined
using predefined functions, for example \Code{Bernoulli(0.3)}
or \Code{Binomial(0.5,8)}, it is assumed to be independent of every other
random variable defined so far.
On the other hand, whenever a random variable is defined using
previously defined random variables, such as \Code{sumXY} above, it is
dependent on them.
Hence, in our example, \Code{X} and \Code{Y} are independent, but
\Code{X} and \Code{sumXY} or \Code{Y} and \Code{sumXY} are not.
{\smart} does not manage dependent random variables in the same expression,
so it issues an error message if it encounters the definition
\begin{lstlisting}
ph int W :=  min(X,sumXY);
\end{lstlisting} 
and sets \Code{W} to \Code{null}.


\subsection{Continuous phase-type random variables}

In the {\smart} language, a random variable with a DCP distribution
has type \Code{ph real}.
Such variables can be defined using predefined functions, or can be built
from other DCP random variables using various operators.
Unlike the discrete case, a \Code{real} cannot be promoted to a \Code{ph real}.
Given
\begin{lstlisting}
ph real X := expo(0.1);
ph real Y := expo(0.2);
\end{lstlisting} 
we can define:
\begin{lstlisting}
ph real sumXY    := X+Y;
ph real prodRX   := 4.0*X;
ph real chooseXY := choose(X:3,Y:7);
ph real minXY    := min(X,Y);
ph real maxXY    := max(X,Y);
ph real geomX    := geom(0.1,X);    // Not implemented currently
\end{lstlisting}
However, definitions such as:
\begin{lstlisting}
ph real diffXY   := X-Y;
ph real sumRX    := 4.0+X;
ph real prodXY   := X*Y;
\end{lstlisting}
are illegal, since DCP distributions are not closed under these
operations.  A correct definition would be
\begin{lstlisting}
rand real diffXY   := X-Y;
rand real sumRX    := 4.0+X;
rand real prodXY   := X*Y;
\end{lstlisting}

We stress that \Code{ph int} and \Code{ph real} values can be freely
intermixed in an expression, but the resulting random variable is
neither a DDP nor a DCP distribution: it has a mixture distribution,
\Code{rand real}, which {\smart} is able to manage only through
Montecarlo simulation.


\subsection{Internal representation of phase-type random variables}

As discussed in Appendix~\ref{SEC:PhaseType}, each DDP or DCP random
variable can be thought of as the time to reach a particular absorbing state
in a discrete-time or continuous time Markov chain.
Indeed, this is the way phase-type random variables are represented internally
in {\smart}.
Thus, care must be taken about the possible combinatorial explosion of
states.
For example, the three random variables \Code{A}, \Code{B}, and \Code{C}
defined by
\begin{lstlisting}
ph int A := equilikely(1,10);
ph int B := equilikely(1,10); 
ph int C := equilikely(1,10); 
\end{lstlisting}
require 11 states each for their representation, one each for the
possible 10 time values, plus one for the final absorbing state $f$.
However,
\begin{lstlisting}
ph int M := min(A,B,C);
\end{lstlisting}
requires $10^3+1 = 1,001$ states for its encoding, since the chain must
contain one state for each possible combination of the time values for the
three random variables \Code{A}, \Code{B}, and \Code{C}, plus, again,
the final absorbing state $f$.

\subsection{Generally-distributed random variables}

{\smart} allows the definition of discrete (integral) or continuous random
variables, having type \Code{rand int} and \Code{rand real}, respectively.
We have seen some examples already, when combining \Code{ph int}
and \Code{ph real} variables through certain operations.
It is also possible to directly define general random variables
using predefined functions that define distributions that
are neither \Code{ph int} nor \Code{ph real}, as in
\begin{lstlisting}
rand real v := uniform(0.1,3.2);
\end{lstlisting}
which defines a random variable \Code{v} with uniform distribution
in the range $[0.1, 3.2]$.

These general random variables can be analyzed using Montecarlo simulation.



%%%%%%%%%%%%%%%%%%%%%%%%%%%%%%%%%%%%%%%%%%%%%%%%%%%%%%%%%%%%%%%%%%%%%%
\section{Stochastic processes} \label{SEC:StochasticProcesses}

Unlike deterministic quantities and random variables,
for which {\smart} provides a small predetermined set of types
unlikely to change in future versions,
the class of stochastic process objects that will eventually be
available in {\smart} is much richer, as it is expected to grow according
to the need of future users.
Two aspects characterize a stochastic process object:
\begin{itemize}
\item
The \emph{type of underlying stochastic process} described by the object.
{\smart} is targeted to discrete-state systems, thus the type of
processes we consider are \cite{Cinlar1975,Kemeny1960}
discrete-time Markov chains,
continuous-time Markov chains,
and more complex process whose stochastic characteristics are determined
by the type of distributions present in the model and their interaction,
such as semi-regenerative processes and generalized semi-Markov processes.
\item
The \emph{formalism} used to describe the object.
Currently, we can explicitly describe discrete-time or continuous-time
Markov chains, or we can describe any of the above processes using
stochastic Petri nets \cite{1994IEEETSE-SPNcharacterization}.
High-level formalisms such as stochastic Petri nets can be exponentially
more compact than explicit descriptions, thus they are the only
practical choice for large modeling applications.
\end{itemize}
In this section, we discuss discrete-time and continuous-time Markov chains.
High-level formalisms are discussed individually in the following chapters.

Note that stochastic process objects can only appear at the top level,
i.e., they cannot be declared or defined within a \Code{for} or \Code{converge}
compound statement.



\begin{comment}
The most important characteristic of a model expressed in a given
formalism is whether it describes a discrete-time stochastic process
$\{X^{[k]} : k \in \Naturals\}$ or a continuous-time stochastic
process $\{X(t) : t \geq 0\}$.  In \smart , two classes of discrete-state Markov
processes are fundamental: discrete-time and continuous-time Markov chains.
In discrete-time Markov chains, the state of the model is observed at each
integer time, while, in continuous-time Markov chains, events occur at
any time, but the amount of time elapsing
between any two events is exponentially distributed.  Such models can
be combined in very natural ways, and are quite general, due to the
concept of {\em phase-type} distributions, which is applicable both to
the discrete and the continuous case.  In both discrete-time and
continuous-time models, the timing of events can have a mass at zero
(immediate events) or at infinity (non-occurring events).
The underlying discrete-time Markov chains and continuous-time Markov chains
can be solved numerically for steady-state or transient analysis.
Alternatively, discrete-event simulation can
be employed for the solution.

However, a model containing both discrete timing (other than zero or
infinity) and continuous timing, defines an underlying process which
can be independent semi-Markovian, semi-Markovian,
semi-regenerative, or generalized semi-Markovian.  The numerical
solution of these processes is sometimes possible depending, among
other factors, on whether a transient or steady-state solution is
desired.  Alternatively, discrete-event simulation is often the method
of choice in these cases, especially for transient analysis.
\begin{release}
Currently, {\smart} does not support discrete-event simulation.
\end{release}
\end{comment}


\subsection{Discrete-time Markov chains}

\begin{figure}  
  \centering
  \includegraphics[scale=0.75]{figures/DTMC1.pdf}
  \caption{An example of a discrete-time Markov chain.}
  \label{FIG:DTMC1}
\end{figure}

The following statements specify the discrete-time Markov chain
\Code{d1} (Fig.~\ref{FIG:DTMC1}), with a parameter \Code{x},
used to specify the value of some of the state-to-state
transition probabilities:
%
\lstinputlisting[firstline=3]{examples/dtmc1a.sm}
%
The definition of the ``value'' of the discrete-time Markov chain
\Code{d1} is performed via a
``block'', that is, a sequence of statements enclosed in braces, used
on the right-hand-side of an assignment.  The statement
\Code{state a,b,c;} declares three ``states''.
Since a state does not carry a value,
the statement has the form of a declaration, but it is a definition.

The two subsequent statements are just expressions.
Since the functions called have type \Code{void}, we use
them because of their side-effect: the first
one specifies the initial state distribution, the second one
specifies the probability of some state-to-state transitions.  
The composition operator ``\Code{:}'' is used to build objects of a composite
type (analogous to the Pascal ``record'' or the C ``struct'').
For example, \Code{a:1.0} associates the real
value \Code{1.0} with the state \Code{a}, while \Code{a:b:1.0}
indicates an arc from state \Code{a} to state \Code{b} with weight \Code{1.0}
(a probability in this case, since we are talking about a discrete time
Markov chain).

Finally, the last three statements, with a familiar syntax, define
\emph{measures}, that is, quantities that {\smart} can compute upon request.
For example, outside the definition of \Code{d1}, the statement
%
\lstinputlisting[firstline=3]{examples/dtmc1b.sm}
%
outputs the nine value pairs corresponding to the steady-state probability
of discrete-time Markov chain \Code{d1} being in state \Code{c},
and to the accumulated time spent in state \Code{a} up to absorption, when
its formal parameter \Code{x} varies from \Code{0.1} to \Code{0.9}
(of course, since \Code{c} is an absorbing state while \Code{a}
and \Code{b} are transient, \Code{d1(v).m1} has value \Code{1}
regardless of the value of \Code{v}).
The expression \Code{avg(tta(is\_absorbed))} evaluates to the average, or mean,
time-to-absorption (often called ``MTTA'').
Note that the function we use, \Code{tta}, has type \Code{ph int},
since it is the time to reach a set of states 
(in this case, the absorbing ones)
in a DTMC;
inside a CTMC model, \Code{tta} instead has type \Code{ph real}.

The selection operator ``\Code{.}'' acts just like in ANSI C, where it
is used to select a field in a structure.  In our case, the ``fields''
are the identifiers declared within the block.
Thus, \Code{m1} denotes a \Code{real} value when it appears within
the block defining the \Code{d1}, but the same quantity must be
referred to as \Code{d1(x:=...).m1} outside that block.

Blocks can only appear at the top level and cannot contain a
\Code{converge} statement or other blocks within them.
However, they can contain \Code{for} statements, which are
extremely useful when defining large repetitive structures.
For example,
%
\lstinputlisting[firstline=3]{examples/dtmc2a.sm}
%
defines a discrete time Markov chain with an unspecified number
\Code{n} of states connected in a circular fashion
(to avoid periodicity, there are two transitions emanating from state
\Code{a[i]}, one to itself and the other to the state \Code{a[i+1]},
or \Code{a[0]} when \Code{i} equals \Code{n-1}, with weight
\Code{1.0} and \Code{9.0}, respectively:
{\smart} normalizes these weights, so that the resulting probabilities are
\Code{0.1} and \Code{0.9}, respectively).
Then, the call
%
\lstinputlisting[firstline=3]{examples/dtmc2b.sm}
%
instantiates the chain \Code{d2} with 10,000 states and computes the
steady-state probability of being in state \Code{a[0]}, which, of course,
equals \Code{0.0001}, by symmetry.

Functions used to define components of a model, such as \Code{init} and
\Code{arcs}, return a type \Code{void} and can be called with multiple
occurrences of the appropriate parameters.
For example,
\begin{lstlisting}
    arcs(a[i]:a[i]:1.0, a[i]:a[mod(i+1,n)]:9.0);
\end{lstlisting}
is exactly equivalent to
\begin{lstlisting}
    arcs(a[i]:a[i]:1.0);
    arcs(a[i]:a[mod(i+1,n)]:9.0);
\end{lstlisting}


Finally, note that the structure of a model can be parameterized in essentially
arbitrary ways by using functions such as \Code{cond} or \Code{case} on
appropriate parameters.
For example
%
\lstinputlisting[firstline=3]{examples/dtmc3.sm}
%
defines either a DTMC where states \Code{a} and \Code{b} are recurrent
(Fig.~\ref{FIG:DTMC3}, left),
or a DTMC where state \Code{a} is transient and \Code{b} is absorbing
(Fig.~\ref{FIG:DTMC3}, right), depending on the value of \Code{connect}.
Thus, \Code{d3(true).m} evaluates to \Code{0.5}, while \Code{d3(false).m}
evaluates to \Code{1.0}.

\begin{figure}
  \centering
  \includegraphics[scale=0.75]{figures/DTMC3.pdf}
  \caption{A parametrized DTMC.}
  \label{FIG:DTMC3}
\end{figure}


\subsection{Continuous-time Markov chains} \label{SEC:CTMC}

\begin{figure}
  \centering
  \includegraphics[scale=0.75]{figures/CTMC1.pdf}
  \caption{An example of a CTMC.}
  \label{FIG:CTMC1}
\end{figure}


The specification of a continuous-time Markov chain is analogous to that
of a discrete-time Markov chain.
The following code defines the continuous-time Markov chain
in Fig.~\ref{FIG:CTMC1}, and defines a measure to compute the probability
of being in state \Code{c} in steady state:
%
\lstinputlisting[firstline=3]{examples/ctmc1a.sm}
%
The compound statement defining a continuous time Markov chain can
contain calls to the functions \Code{init} and \Code{arcs},
defined as for the discrete time case, with two differences:
the arc weights represent rates, not probabilities,
hence they are not normalized, and
arcs from a state to itself are ignored, but a warning is issued.

The measure defined, \Code{m1}, is set to the steady-state expected reward rate
when the reward rates assigned to states \Code{a}, \Code{b}, and \Code{c}
are \Code{2}, \Code{1}, and \Code{0}, respectively.
This value can then be computed for various
values of \Code{lambda} using the \Code{for}-loop statement
%
\lstinputlisting[firstline=3]{examples/ctmc1b.sm}
%
which results in the following output:
\begin{lstlisting}
lambda=0.03   reward= 1.6692913
lambda=0.05   reward= 1.3333333
lambda=0.07   reward= 1.0410959
\end{lstlisting}


\subsection{Measures}

As we have seen, the way to obtain information out of a
model is to define a measure, and then ask for its value outside the model.
We have already seen functions, such as \Code{prob\_ss} and \Code{avg\_ss},
that can be used to query timing or stochastic properties;
this section discusses in detail all the functions available for this purpose.
Functions that can be used to query state-space and temporal logic properties
are instead discussed in Chapters \ref{SEC:SSGen} and \ref{SEC:ModelChecking}.

Let $\{X(t):t\geq 0\}$ represent the stochastic process describing
some time-dependent \Code{bool} condition \Code{x},
such the expression \Code{in\_state(c)}
used to define measure \Code{m1} in the DTMC model \Code{d1}.
Then:
\begin{itemize}
\item
\Code{prob\_ss(x)} is
the probability that condition \Code{x} is \Code{true} in steady-state:
$$\lim_{t\ \rightarrow \infty} \Pr\{X(t)\}.$$
\item
\Code{prob\_at(x,100)} is
the probability that condition \Code{x} is \Code{true} at time 100:
$$\Pr\{X(100)\}.$$
\item
\Code{prob\_acc(x,100,300)} is
the accumulated probability (i.e., the expected amount of time) that condition
\Code{x} is \Code{true} during the interval [100,300]:
$$\int_{100}^{300} \Pr\{X(t)\} dt.$$
\end{itemize}

Let $\{Y(t):t\geq 0\}$ represent the stochastic process describing
some time-dependent \Code{int} or \Code{real} quantity \Code{y},
such as the expression \Code{cond(in\_state(a),2,cond(in\_state(b),1,0))}
used to define measure \Code{m1} in the CTMC model \Code{c1}.
Then:
\begin{itemize}
\item \Code{avg\_ss(y)} is
the expected value of \Code{y} in steady-state:
$$\lim_{t\ \rightarrow \infty} E[Y(t)].$$
\item \Code{avg\_at(y,100)} is
the expected value of \Code{y} at time 100:
$$E[Y(100)].$$
\item \Code{avg\_acc(y,100,300)} is
the expected accumulated value of \Code{y} during the interval [100,300]:
$$\int_{100}^{300} E[Y(t)] dt.$$
\end{itemize}




\begin{comment}
\subsection{Independent semi-Markov processes}

The arc components in an independent semi-Markov process are probabilities
(possibly to be normalized), as in a discrete-time Markov chain, but the
holding time in each a state can be arbitrarily
distributed, instead of being deterministically equal one:
\begin{lstlisting}ismp i1 := {
  state a,b,c;
  init(a:1);
  arcs(a:b:1,b:a:0.9,b:c:0.1);
  holding(a:const(3),b:expo(20.0));
  real m1 := prob_at(in_state(c),100);
};
\end{lstlisting}
The compound statement defining an independent semi-Markov process can
contain at most one call to the functions \Code{init}, \Code{arcs},
and \Code{assert}, defined as for a discrete-time Markov chain,
and to \Code{holding}, defined as follows:
\begin{itemize}
\item
\Code{void holding(state:$x$ s1:h1, state:$x$ s2:h2, ...);}\\
where $x$ is \Code{ph int}, \Code{ph real}, \Code{rand int}, or
\Code{rand real}.
Sets the holding time for state \Code{s} to \Code{h}.
If a state has no outgoing arcs, no holding time needs to be specified,
the state is considered absorbing with a holding time
equal \Code{const(infinity)}.
%If the holding time of a non-absorbing state is not specified,
%the default \Code{const(0)} is assumed.
%This makes the corresponding state vanishing.
\end{itemize}

\subsection{Semi-Markov processes}

The most general state-space process that can be explicitly specified
in the {\smart} language is a semi-Markov process.
In this case, a distribution is associated with each arc from $i$ to $j$.
There are two ways to define a semi-Markov process:
\begin{description}
\item{\bf Race}:
given that the current state is $i$, specify the distribution $F_{i,j}(t)$
for the time required to go from
$i$ to $j$ in isolation, that is, ignoring the other arcs leaving from $i$.
The actual semantic is then that of a race among all the arcs leaving $i$,
hence the holding time for $i$ is implicitly assigned the distribution of the
minimum of these times, over all the arcs leaving state $i$.
\item{\bf Preselection}:
choose according to a specified probability $p_{i,j}$ the next state $j$,
given that the current state is $i$.
Then, sample the distribution $F_{i,j}(t)$ to select the time required
to go from $i$ to $j$ in isolation.
\end{description}
While either method can specify more general stochastic processes than
independent semi-Markov processes,
the former method is not as general as the latter.
Fig.~\ref{FIG:semiMarkov} shows the equivalent ways to specify simpler
processes (discrete-time Markov chains, continuous-time Markov chains,
or independent semi-Markov processes) as semi-Markov processes,
with either method.

\begin{figure}
\FIG{semiMarkov}
\caption{Equivalent semi-Markov processes.}
\label{FIG:semiMarkov}
\end{figure}

For example, the following specifies a preselection semi-Markov process
\Code{s1} exactly equivalent to the independent semi-Markov process
\Code{i1} specified in the previous section:
\begin{lstlisting}smp s1 := {
  state a,b,c;
  init(a:1);
  arcs(a:b:1:const(3),b:a:0.9:expo(20.0),b:c:0.1:expo(20.0));
  real m1 := prob(at(in_state(c),100));
};
\end{lstlisting}
Alternatively, we could have used a race specification for the arcs:
\begin{lstlisting}
  arcs(a:b:const(3),b:a:expo(18.0),b:c:expo(2.0))
\end{lstlisting}

The race specification is often more convenient and more natural.
However, it does not specify how to ``break the tie'', and this might be
a problem in the presence of discrete distributions.
\IGNORE{Is this a real limitation? Need to check (Kulkarni?)}

The compound statement defining a semi-Markov process can contain at most
one call to the functions \Code{init} and \Code{assert}, defined as
for a discrete-time Markov chain, and \Code{arcs},
an overloaded function:
\begin{itemize}
\item
\Code{void arcs(state:state:$x$ s1:t1:v1, state:state:$x$ s2:t2:v2, ...);}\\
where $x$ is \Code{ph int}, \Code{ph real}, \Code{rand int}, or
\Code{rand real}.
Sets the distribution for the time to go from state \Code{s}$i$ to state
\Code{t}$i$ to \Code{v}$i$, according to a race semantic.
\item
\Code{void arcs(state:state:real:$x$ s1:t1:p1:v1, state:state:real:$x$ s2:t2:p2:v2, ...);}\\
where $x$ is \Code{ph int}, \Code{ph real}, \Code{rand int}, or
\Code{rand real}.
Sets the probability and the distribution for the time to go from state
\Code{s}$i$ to state \Code{t}$i$ to \Code{p}$i$ and \Code{v}$i$,
respectively, according to a preselection semantic.
\end{itemize}

\end{comment}


\TBD{Discuss the ``model'' type.  Arrays of models.}
