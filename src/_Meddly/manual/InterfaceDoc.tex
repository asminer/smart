
\documentclass[12pt]{article}

\usepackage{fullpage}
\usepackage{color}

\newcommand{\TBD}[1]{{\color{blue}{\bf TBD:} #1}}


\begin{document}

\section{C-style interface to the MDD Library}

\subsection{Forest Operations}

\subsubsection{int create\_forest()}

Initializes the mdd\_forest.
If the forest is already initialized, do nothing.
Return 0 on success, nonzero on failure.

\TBD{
For now assume there is only one forest.
If we want to allow multiple forests in the future,
then we will return a handle here and all other operations will
also pass the forest handle.
}

\subsubsection{void destroy\_forest()}

Releases the mdd\_forest's resources and discards it.
Does nothing if the forest is not yet initialized or already destroyed.

\subsection{MDD Node Operations}

Note: All of these should cleanly return an appropriate error if the 
forest has not been initialized.

\subsubsection{mdd\_node create\_node(int level, int size)}

\begin{itemize}
\item
Creates and returns a new mdd\_node of specifed size and level. 
\item
This node does not yet exist in the unique table (unreduced) and 
its down pointers can therefore be modified.
\item
Incoming count to this node is 1 and compute table count is 0.
\item
Returns 0 on error.
\end{itemize}

\subsubsection{const mdd\_node* get\_down\_ptrs\_read\_only(mdd\_node a\_node)}

\begin{itemize}
\item
Returns the down pointers for the specified mdd\_node.
\item
These down pointers cannot be modified.
\item
Returns NULL on error.
\end{itemize}

\subsubsection{mdd\_node* get\_down\_ptrs(mdd\_node a\_node)}

\begin{itemize}
\item
Returns the down pointers for the specified mdd\_node.
\item
These down pointers can be modified.
\item
Returns NULL on error.
\end{itemize}

\TBD{
Q: Can this function be called once the node has been reduced? If it can, won't the user's ability to modify the down pointers violate the property that a reduced node should not be modified?

No, if the node is already reduced, NULL should be returned.
}

\subsubsection{int reduce(mdd\_node *a\_node)}

\begin{itemize}
\item
Checks if a\_node is already reduced. If so, returns 0 for success.
\item
Otherwise, checks if a duplicate of a\_node exists in the unique table. If it does exist:
\begin{itemize}
\item
a\_node is unlinked and set to point to the duplicate node; the duplicate
node's incoming count is incremented.
\item
Otherwise (no duplicate found), a\_node is added to the unique table.
\end{itemize}
In both cases, return 0 on success.
\end{itemize}

\TBD{
Q: How do we know if a node is already reduced (other than by searching the
unique table for a duplicate)? Is there a flag that is checked?

Recall that a node includes a pointer (integer) for use in the uniqueness
table; this can be set to bogus values for temporary (unreduced) nodes and
deleted nodes.
}

\subsubsection{int unlink\_node(mdd\_node a\_node)}

\begin{itemize}
\item
If incoming count is already 0, return nonzero.
\item
Otherwise, decrement incoming count for a\_node. If incoming count is now
zero, recycle node.
\item
Returns 0 for sucess.
\end{itemize}

\subsubsection{int link\_node(mdd\_node a\_node)}

\begin{itemize}
\item
Increment incoming link count for a\_node.
\item
Returns 0 for sucess.
\end{itemize}

\subsubsection{int show\_node(FILE *output\_stream, mdd\_node a\_node)}

Prints a representation of a\_node on output\_stream.

\noindent
Sample: node: 7  level: 3  in: 5   down: [...]

\subsubsection{int show\_all\_nodes(FILE *output\_stream)}

Prints a representation of all active mdd\_nodes on output\_stream. Ordering: First by node level and then then by node number.

\subsection{Garbage Collection}

\subsubsection{int invoke\_gc()}

Forces garbage collection (instead of waiting for the garbage collector to act on it own).

\subsection{Statistics}

\subsubsection{int report(FILE *output\_stream)}

Prints statistics collected during program execution on the output\_stream.

\end{document}
